	% 代码分析:模块功能、涉及到的类、类关系、数据结构及关键代码等;
	% 任务要求,设计任务要求;
	% 设计:详细的设计方案,相关的数据结构、算法描述,可采用伪代码等形式化描述
	% 实现:修改哪些类、如何修改、为什么修改等;
	% 测试:测试用例,测试结果及结果分析,测试运行界面等;
	% 调试:调试方法,遇到的问题及解决方案等;
	% 结论与展望:完成的主要工作、收获、进一步的工作,建议、体会、心得等;

\section{Experiment 7: SVM}

\subsection{SVM}

支持向量机是一种监督学习模型,广泛应用于分类和回归问题。其核心思想是通过一个超平面将不同类别的样本数据进行分隔,从而实现分类。在二分类问题中,假设数据集为\(\{(x_i, y_i)\}\),其中\(x_i \in \mathbb{R}^n\)是特征向量,\(y_i \in \{-1, +1\}\)是类别标签。SVM的目标是找到一个决策超平面,该平面可以将两类数据分开,并且尽可能使得两类数据之间的间隔(即“边界”)最大化。

一个超平面可以表示为

\[
w \cdot x + b = 0
\]

其中,\(w\)是法向量,决定了超平面的方向,\(b\)是偏置,控制超平面的位置。

SVM的目标是最大化两类数据点到超平面的间隔。这个间隔可以通过支持向量(离超平面最近的样本点)来定义。为了最大化间隔,我们需要优化以下目标:

\[
\min_{w, b} \frac{1}{2} \|w\|^2
\]

同时满足约束条件:

\[
y_i (w \cdot x_i + b) \geq 1, \quad \forall i
\]

这意味着每个样本点都被正确分类,并且位于超平面的一侧。


首先读取数据:

\begin{lstlisting}[language=matlab,title={main.m}]
data1 = load('training_1.txt');
X1 = data1(:, 1:2);
y1 = data1(:, 3);

test1 = load('test_1.txt');
X_test1 = test1(:, 1:2);
y_test1 = test1(:, 3);
\end{lstlisting}

SVM 的训练:

\begin{lstlisting}[language=matlab,title={trainSVM.m}]
function [C, w, b] = trainSVM(X, y, C)
    H = (y * y') .* (X * X');
    f = -ones(size(y));
    A = [];
    b = [];
    Aeq = y';
    beq = 0;
    lb = zeros(size(y));
    ub = 0.00001 * ones(size(y));
    
    alpha = quadprog(H, f, A, b, Aeq, beq, lb, ub);

    w = sum((alpha .* y) .* X)';
    b = mean(y - X * w);
end
\end{lstlisting}

SVM 的测试:

\begin{lstlisting}[language=matlab,title={testSVM.m}]
function misclassified_fraction = testSVM(X_test, y_test, w, b)
    predictions = X_test * w + b;
    predicted_labels = sign(predictions);
    
    misclassified = sum(predicted_labels ~= y_test) / length(y_test);
    misclassified_fraction = misclassified;
end
\end{lstlisting}

不难画出训练集与测试集上的图像:

\begin{lstlisting}[language=matlab,title={plotDecisionBoundary.m}]
function plotDecisionBoundary(X, y, w, b, title_text)
    % Function to plot decision boundary and data points
    figure;
    x1_range = linspace(min(X(:, 1)), max(X(:, 1)), 100);
    x2_range = linspace(min(X(:, 2)), max(X(:, 2)), 100);
    [x1_grid, x2_grid] = meshgrid(x1_range, x2_range);
    X_grid = [x1_grid(:), x2_grid(:)];
    predictions_grid = X_grid * w + b;
    
    contour(x1_grid, x2_grid, reshape(predictions_grid, size(x1_grid)), [0, 0], 'k', 'LineWidth', 2);
    hold on;
    scatter(X(:, 1), X(:, 2), 50, y, 'filled');
    title(title_text);
    xlabel('x1');
    ylabel('x2');
    legend('Decision Boundary', 'Class 1', 'Class -1');
end
\end{lstlisting}

\begin{figure}[H]
    \centering
    \includesvg[width=10cm]{imgs/7-1-1.svg}
\end{figure} 

\begin{figure}[H]
    \centering
    \includesvg[width=10cm]{imgs/7-1-2.svg}
\end{figure} 

\subsection{Handwritten Digit Recognition}

首先参考给出的 strimage.m 读入文件:

\begin{lstlisting}[language=matlab,title={main.m}]
acc = [];
X = [];
y = [];

fidin = fopen('train-01-images.svm');
i = 1;
apres = [];
while ~feof(fidin)
    tline = fgetl(fidin);
    apres{i} = tline;
    i = i + 1;
end

for i = 1:12665
    a = char(apres(i));
    lena = size(a, 2);
    xy = sscanf(a(4:lena), '%d:%d');
    lenxy = size(xy, 1);
    
    grid = zeros(1, 784);
    for j = 2:2:lenxy
        if (xy(j) <= 0)
            break
        end
        grid(xy(j-1)) = xy(j) * 100 / 255;  % Scale pixel value
    end
    
    X = [X; grid];
    if startsWith(a, '+')
        y = [y, 1];
    elseif startsWith(a, '-')
        y = [y, -1];
    end
end
fclose(fidin);

Cval = [0.0001, 0.001, 1, 10, 100, 1000, 10000, 1000000, 100000000];
\end{lstlisting}

训练代码为:

\begin{lstlisting}[language=matlab,title={main.m}]
for i = 1:length(Cval)
    C = Cval(i);
    [w, b] = trainSVM(X, y', C);
    
    train_error = computeError(X, y', w, b);
    fprintf('C = %.7f, Training Error: %.4f\n', C, train_error);
    
    fidin = fopen('test-01-images.svm');
    apres = [];
    while ~feof(fidin)
        tline = fgetl(fidin);
        apres{i} = tline;
        i = i + 1;
    end
    X_test = [];
    y_test = [];
    
    for j = 1:2115
        a = char(apres(j));
        lena = size(a, 2);
        xy = sscanf(a(4:lena), '%d:%d');
        lenxy = size(xy, 1);
        
        grid = zeros(1, 784);  % Initialize feature vector for test image
        for k = 2:2:lenxy
            if (xy(k) <= 0)
                break
            end
            grid(xy(k-1)) = xy(k) * 100 / 255;  % Scale pixel value
        end
        
        X_test = [X_test; grid];
        if startsWith(a, '+')
            y_test = [y_test, 1];
        elseif startsWith(a, '-')
            y_test = [y_test, -1];
        end
    end
    fclose(fidin);
    
    test_error = computeError(X_test, y_test', w, b);
    fprintf('C = %.7f, Test Error: %.4f\n', C, test_error);

    acc = [acc, test_error];
end
\end{lstlisting}

可以得到收敛结果:

\begin{lstlisting}[language=matlab,title={main.m}]
Optimization completed: The relative dual feasibility, 6.508655e-12,
is less than options.OptimalityTolerance = 1.000000e-08, the complementarity measure,
1.205339e-09, is less than options.OptimalityTolerance, and the relative maximum constraint
violation, 1.100637e-18, is less than options.ConstraintTolerance = 1.000000e-08.
\end{lstlisting}

可以找出无法分类的图像:

\begin{lstlisting}[language=matlab,title={main.m}]
misclassified_idx = findMisclassified(X_train, y_train, w0, b0);
visualizeMisclassified(X_train, misclassified_idx);
\end{lstlisting}

\begin{lstlisting}[language=matlab,title={findMisclassified.m}]
function misclassified_idx = findMisclassified(X, y, w, b)
    % Function to find the index of a misclassified example
    predictions = X * w + b;
    predicted_labels = sign(predictions);
    misclassified_idx = find(predicted_labels ~= y, 1, 'first'); % Find the first misclassified example
end
\end{lstlisting}

\begin{lstlisting}[language=matlab,title={visualizeMisclassified.m}]
function visualizeMisclassified(X, idx)
    % Function to visualize a misclassified example
    misclassified_image = reshape(X(idx, :), [28, 28]); % Reshape to 28x28 image
    figure;
    imshow(misclassified_image, []);
    title('Misclassified Image');
end
\end{lstlisting}

\begin{figure}[H]
    \centering
    \includesvg[width=5cm]{imgs/7-2-1.svg}
\end{figure} 


\subsection{Non-Linear SVM}

进行升维:

\begin{lstlisting}[language=matlab,title={trainNonLinearSVM.m}]
            K(i, j) = exp(-r * norm(X(i, :) - X(j, :))^2);
\end{lstlisting}

不难写出训练函数:

\begin{lstlisting}[language=matlab,title={trainNonLinearSVM.m}]
function [alpha, w, b] = trainNonLinearSVM(X, y, C, r)
    % Train a non-linear SVM using the RBF kernel with specified regularization parameter C and gamma (r)
    n = size(X, 1);
    
    % Compute the kernel matrix K
    K = zeros(n, n);
    for i = 1:n
        for j = 1:n
            K(i, j) = exp(-r * norm(X(i, :) - X(j, :))^2);
        end
    end
    
    % Set up the optimization problem for SVM
    H = (y * y') .* K;
    f = -ones(n, 1);
    A = [];
    b = [];
    Aeq = y';
    beq = 0;
    lb = zeros(n, 1);
    ub = C * ones(n, 1);
    
    % Solve the quadratic programming problem
    alpha = quadprog(H, f, A, b, Aeq, beq, lb, ub);
    
    % Calculate the weight vector w and bias b
    w = sum((alpha .* y) .* X, 1)';
    b = mean(y - K * (alpha .* y));
end
\end{lstlisting}


\begin{figure}[H]
    \centering
    \includesvg[width=18cm]{imgs/7-3-1.svg}
\end{figure} 

\subsection{完整代码}

\begin{lstlisting}[language=matlab,title={SVM - main.m}]
clear;
data1 = load('training_1.txt');
X1 = data1(:, 1:2);
y1 = data1(:, 3);
test1 = load('test_1.txt');
X_test1 = test1(:, 1:2);
y_test1 = test1(:, 3);
[C, w, b] = trainSVM(X1, y1, 0.00001); % Regularization term C
plotDecisionBoundary(X1, y1, w, b, 'Dataset 1: Decision Boundary');
misclassified_fraction = testSVM(X_test1, y_test1, w, b);
fprintf('Fraction of misclassified examples for Dataset 1: %.2f\n', misclassified_fraction);
data2 = load('training_2.txt');
X2 = data2(:, 1:2);
y2 = data2(:, 3);
test2 = load('test_2.txt');
X_test2 = test2(:, 1:2);
y_test2 = test2(:, 3);
[C2, w2, b2] = trainSVM(X2, y2, 0.00001);
plotDecisionBoundary(X2, y2, w2, b2, 'Dataset 2: Decision Boundary');
misclassified_fraction2 = testSVM(X_test2, y_test2, w2, b2);
fprintf('Fraction of misclassified examples for Dataset 2: %.2f\n', misclassified_fraction2);
\end{lstlisting}


\begin{lstlisting}[language=matlab,title={plotDecisionBoundary.m}]
function plotDecisionBoundary(X, y, alpha, r, C)
    % This function plots the decision boundary for a non-linear SVM with RBF kernel.
    % X: Training data (n x 2)
    % y: Labels (n x 1)
    % alpha: Coefficients from SVM solution (n x 1)
    % r: Gamma parameter for RBF kernel
    % C: Regularization parameter

    % Create a grid for plotting the decision boundary
    [X1, X2] = meshgrid(linspace(min(X(:, 1)), max(X(:, 1)), 100), ...
                        linspace(min(X(:, 2)), max(X(:, 2)), 100));
    grid_points = [X1(:), X2(:)];
    
    % Compute the kernel matrix for grid points
    K_test = zeros(size(grid_points, 1), size(X, 1));
    for i = 1:size(grid_points, 1)
        for j = 1:size(X, 1)
            K_test(i, j) = exp(-r * norm(grid_points(i, :) - X(j, :))^2);
        end
    end
    
    % Compute decision values
    decisions = K_test * (alpha .* y);
    
    % Plot decision boundary
    contour(X1, X2, reshape(decisions, size(X1)), [0, 0], 'LineWidth', 2);
    hold on;
    
    % Scatter plot for training data
    scatter(X(:, 1), X(:, 2), 50, y, 'filled');
    hold off;
    
    % Labels
    xlabel('x_1');
    ylabel('x_2');
    title(['Decision Boundary: C = ', num2str(C), ', r = ', num2str(r)]);
    colorbar;
end
\end{lstlisting}


\begin{lstlisting}[language=matlab,title={loadImageData.m}]
function [X, y] = loadImageData(filename)
    % 读取SVM格式的图像数据(每个图像数据为一行)
    fidin = fopen(filename); % 打开文件
    i = 1;
    apres = {};

    % 读取每一行数据
    while ~feof(fidin)
        tline = fgetl(fidin); % 从文件读取一行
        apres{i} = tline;
        i = i + 1;
    end

    fclose(fidin);

    % 初始化数据矩阵
    X = [];
    y = [];
    
    % 遍历读取的每一行
    for n = 1:length(apres)
        a = char(apres{n});
        
        % 获取标签
        label = str2double(a(1));
        y = [y; label];
        
        % 获取特征数据 (每个图像为784维)
        lena = length(a);
        xy = sscanf(a(4:lena), '%d:%d'); % 解析特征
        
        % 初始化一个784维的特征向量
        grid = zeros(1, 784); 
        lenxy = length(xy) / 2;
        
        % 填充特征数据到特征向量
        for i = 1:2:lenxy
            index = xy(i); % 特征索引
            value = xy(i+1); % 特征值
            if index > 0
                grid(index) = value * 100 / 255; % 归一化处理
            end
        end
        
        % 将特征向量添加到特征矩阵
        X = [X; grid];
    end
    
    % 转换为double类型
    X = double(X);
    y = double(y);
end
\end{lstlisting}


\begin{lstlisting}[language=matlab,title={trainNonLinearSVM.m}]
[X_train, y_train] = loadData('training_3.text');

C_values = [1, 100, 1000];
r_values = [0.1, 10, 100];

figure;
hold on;

for i = 1:length(C_values)
    for j = 1:length(r_values)
        C = C_values(i);
        r = r_values(j);
        
        [alpha, w, b] = trainNonLinearSVM(X_train, y_train, C, r);
        
        subplot(length(C_values), length(r_values), (i-1) * length(r_values) + j);
        plotDecisionBoundary(X_train, y_train, alpha, r, C);
        
        title(['C = ' num2str(C) ', r = ' num2str(r)]);
    end
end

xlabel('x_1');
ylabel('x_2');
sgtitle('Non-linear SVM Decision Boundaries');
grid on;
hold off;
\end{lstlisting}