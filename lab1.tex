	% 代码分析:模块功能、涉及到的类、类关系、数据结构及关键代码等;
	% 任务要求,设计任务要求;
	% 设计:详细的设计方案,相关的数据结构、算法描述,可采用伪代码等形式化描述
	% 实现:修改哪些类、如何修改、为什么修改等;
	% 测试:测试用例,测试结果及结果分析,测试运行界面等;
	% 调试:调试方法,遇到的问题及解决方案等;
	% 结论与展望:完成的主要工作、收获、进一步的工作,建议、体会、心得等;

\section{Experiment 1: Linear Regression}
\subsection{2D Linear Regression}

首先读入数据到 matlab:

\begin{lstlisting}[language=matlab,title={matlab load}]
x = load('ex1x.dat');
y = load('ex1y.dat');
\end{lstlisting}

使用 plot 函数可以进行绘制:

\begin{lstlisting}[language=matlab,title={matlab figure}]
figure % open a new figure window
plot(x,y,'o');
ylabel('Height in meters')
xlabel('Age in years')
\end{lstlisting}

得到的图像为:

\begin{figure}[H]
    \centering
    \includesvg[width=12cm]{imgs/1.svg}
\end{figure} 

将 $x$ 增加偏置项,这样就可以进行线性回归:

\begin{lstlisting}[language=matlab,title={matlab regression init}]
m=length(y); % store the number of training examples
x=[ones(m,1),x]; % Add a column of ones to x
\end{lstlisting}

线性回归的计算模型为:

\[
h_{\theta}(x)=\theta^{T} x=\sum_{i=0}^{n} \theta_{i} x_{i}
\]

线性回归梯度下降的公式为:

\[
\theta_{j}:=\theta_{j}-\alpha \frac{1}{m} \sum_{i=1}^{m}\left(h_{\theta}\left(x^{(i)}\right)-y^{(i)}\right) x_{j}^{(i)}
\]

\begin{lstlisting}[language=matlab,title={Linear Regression}]
alpha=0.07;
iterations=1500;
theta=zeros(2,1);

for iter=1:iterations
    h=x*theta;
    error=h-y;
    theta=theta-(alpha/m)*(x'*error);
end

disp(theta);
\end{lstlisting}

输出 0.7502 与 0.0639,也就是 $\theta_1=0.7502, \theta_2=0.0639$,可以在原图像基础上绘制回归直线:

\begin{lstlisting}[language=matlab,title={figure regression}]
hold on
plot(x(:,2),x*theta,'-')
legend('Training data','Linear regression')
\end{lstlisting}

得到图像:

\begin{figure}[H]
    \centering
    \includesvg[width=12cm]{imgs/1-2.svg}
\end{figure} 

计算 3.5 岁与 7 岁同学的身高,直接代入得:

\begin{lstlisting}[language=matlab,title={linear predict}]
disp([1,3.5]*theta)
disp([1,7]*theta)
\end{lstlisting}

得到 0.9737 与 1.1973。

% \begin{figure}[H]
%     \centering
%     \includegraphics[width = 15cm]{imgs/hw1_3.jpg}
% \end{figure} 

% \begin{figure}[H]
%     \centering
%     \includegraphics[width = 15cm]{imgs/hw1_4.jpg}
% \end{figure} 

\subsection{Understanding $J(\theta)$}

由 $J(\theta )=\frac{1}{2m}\sum _{i=1}^n\left (h_\theta (x_i)-y_i\right ) ^2$,不难写出代码:

\begin{lstlisting}[language=matlab,title={calculate $J(\theta)$}]
theta1_vals=linspace(-10,10,100);
theta2_vals=linspace(-10,10,100);
[J,theta1,theta2]=meshgrid(theta1_vals,theta2_vals);
J_vals=zeros(size(theta1));

for i=1:size(theta1,1)
    for j=1:size(theta1,2)
        theta_temp=[theta1(i,j);theta2(i,j)];
        h=x*theta_temp;
        J_vals(i,j)=(1/(2*m))*sum((h-y).^2);
    end
end
\end{lstlisting}


\begin{figure}[H]
    \centering
    \includesvg[width=12cm]{imgs/1-3.svg}
\end{figure} 

不太直观,实际上还可以绘制梯度下降中迭代的“路径”:

\begin{figure}[H]
    \centering
    \includesvg[width=12cm]{imgs/1-4.svg}
\end{figure} 

\begin{figure}[H]
    \centering
    \includesvg[width=12cm]{imgs/1-5.svg}
\end{figure} 

\subsection{Programming}

绘制路径部分的完整代码如下:

\begin{lstlisting}[language=matlab,title={figure surface \& contour plot}]
x=load('ex1x.dat');
y=load('ex1y.dat');
m=length(y);
x=[ones(m,1),x];
theta1_vals=linspace(-10,10,50);
theta2_vals=linspace(-10,10,50);
[theta1,theta2]=meshgrid(theta1_vals,theta2_vals);
J_vals=zeros(size(theta1));

for i=1:size(theta1,1)
    for j=1:size(theta1,2)
        theta_temp=[theta1(i,j);theta2(i,j)];
        h=x*theta_temp;
        J_vals(i,j)=(1/(2*m))*sum((h-y).^2);
    end
end

alpha=0.07;
iterations=1000;
theta=[6;7.5];
J_history=zeros(iterations,1);
theta_history=zeros(iterations,2);

for iter=1:iterations
    h=x*theta;
    error=h-y;
    theta=theta-(alpha/m)*(x'*error);
    J_history(iter)=(1/(2*m))*sum((h-y).^2);
    theta_history(iter,:)=theta';
end

figure;
surf(theta1,theta2,J_vals);
hold on;

view(30,30);
shading interp;
colorbar;

final_cost=(1/(2*m))*sum((x*theta-y).^2);
plot3(theta(1),theta(2),final_cost,'r*','MarkerSize',10);
xlabel('\theta_1');
ylabel('\theta_2');
zlabel('J(\theta)');
title('Surface plot of J(\theta) for Linear Regression');

for i=1:10:iterations
    if i+10<=iterations
        plot3(theta_history(i,1),theta_history(i,2),...
              (1/(2*m))*sum((x*theta_history(i,:)'-y).^2),'bo','MarkerSize',5);
        quiver3(theta_history(i,1),theta_history(i,2),...
                 (1/(2*m))*sum((x*theta_history(i,:)'-y).^2),...
                 theta_history(i+10,1)-theta_history(i,1),...
                 theta_history(i+10,2)-theta_history(i,2),...
                 0,'k','LineWidth',2,'AutoScale','off');
    end
end

gradient=(1/m)*(x'*(h-y));
quiver3(theta(1),theta(2),final_cost,gradient(1),gradient(2),0,'k','LineWidth',2,'AutoScale','off');

text(theta(1),theta(2),final_cost,sprintf('  (%.2f,%.2f)',theta(1),theta(2)),'Color','red');

hold off;

print('surface_plot.svg','-dsvg');

figure;
contour(theta1,theta2,J_vals,50);
hold on;

plot(theta(1),theta(2),'r*','MarkerSize',10);
xlabel('\theta_1');
ylabel('\theta_2');
title('Contour plot of J(\theta) for Linear Regression');
colorbar;

num_arrows=10;
for i=1:num_arrows:iterations
    if i+num_arrows<=iterations
        quiver(theta_history(i,1),theta_history(i,2),...
               theta_history(i+num_arrows,1)-theta_history(i,1),...
               theta_history(i+num_arrows,2)-theta_history(i,2),...
               'k','LineWidth',2,'AutoScale','off');
    end
end

hold off;
\end{lstlisting}