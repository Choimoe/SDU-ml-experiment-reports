\documentclass[UTF8]{ctexart}
\usepackage{graphicx}
\usepackage{float}
\usepackage{fancyhdr}
\usepackage{listings}
\usepackage{xeCJK} 
\usepackage{authblk} 
\usepackage{geometry}
\usepackage{hyperref}
\usepackage{amsmath, amsthm, amssymb, amsfonts}
\usepackage{thmtools}
\usepackage{ctex}
\usepackage{graphicx}
\usepackage{setspace}
\usepackage{geometry}
\usepackage{float}
\usepackage{hyperref}
\usepackage[utf8]{inputenc}
\usepackage[english]{babel}
\usepackage{framed}
\usepackage[dvipsnames]{xcolor}
\usepackage{tcolorbox}
\usepackage{svg}
\usepackage{algorithm}
\usepackage{ctex}
\usepackage{amsmath}
\usepackage{booktabs}
\usepackage{algorithmic}
\usepackage{array}
\usepackage{tikz}
\usetikzlibrary{shapes.geometric, arrows.meta, positioning}


\tikzset{
    block/.style={rectangle, draw, fill=blue!20, text width=5em, text centered, rounded corners, minimum height=4em},
    line/.style={draw, -{Latex[]}},
    cloud/.style={draw, ellipse,fill=red!20, minimum height=2em},
    crypto/.style={rectangle, draw, fill=red!50, text width=5em, text centered, minimum height=2em},
    arch/.style={rectangle, draw, fill=yellow!50, text width=2em, text centered, minimum height=2em},
    kernel/.style={rectangle, draw, fill=green!50, text width=10em, text centered, minimum height=2em},
    exception/.style={draw, -{Latex[width=2pt]}, line width=2pt, color=red}
}

\colorlet{LightGray}{White!90!Periwinkle}
\colorlet{LightOrange}{Orange!15}
\colorlet{LightGreen}{Green!15}
\bibliographystyle{plain}
\geometry{a4paper}
\geometry{left=2cm} 
\geometry{right=2cm} 
\geometry{top=3cm} 
\geometry{bottom=3cm} 
\pagestyle{fancy}
\fancyhf{}
\newcommand{\HRule}[1]{\rule{\linewidth}{#1}}

\declaretheoremstyle[name=Theorem,]{thmsty}
\declaretheorem[style=thmsty,numberwithin=section]{theorem}
\tcolorboxenvironment{theorem}{colback=LightGray}

\setstretch{1.2}
\geometry{
    textheight=9in,
    textwidth=5.5in,
    top=1in,
    headheight=12pt,
    headsep=25pt,
    footskip=30pt
}

\lstset{
    numbers=left, 
    numberstyle= \tiny, 
    keywordstyle= \color{ blue!70},%设置关键字颜色
    commentstyle= \color{red!50!green!50!blue!50}, %设置注释颜色
    frame=shadowbox, % 阴影效果
    rulesepcolor= \color{ red!20!green!20!blue!20} ,
    escapeinside=``, % 英文分号中可写入中文
    xleftmargin=2em, %距离左边界2em
    aboveskip=1em,
    framexleftmargin=2em,
    basicstyle=\ttfamily,
    columns=fullflexible,%可以自动换行
    linewidth=1\linewidth, %设置代码块与行同宽
    breaklines=true,%在单词边界处换行。
    showstringspaces=false, %去掉空格时产生的下划的空格标志, 设置为true则出现
    breakatwhitespace=ture,%可以在空格处换行
    escapechar=`%设置转义字符为反引号
}
\rhead{Choimoe@SDU}
\lhead{机器学习实验报告}
\rfoot{\thepage}
\begin{document}
\title{ \normalsize \textsc{}
		\\ [2.0cm]
		\HRule{2.0pt} \\
		\LARGE \textbf{{机器学习实验报告}
		\HRule{2.0pt} \\ [0.6cm] {————Experiment 4: Regularized Regression} \vspace*{10\baselineskip}}
		}
\date{}
\author{Choimoe \\ 
		CS \\
		SDU \\
		2024.12.16}

\maketitle

\newpage
\tableofcontents

\newpage
	% 代码分析:模块功能、涉及到的类、类关系、数据结构及关键代码等;
	% 任务要求,设计任务要求;
	% 设计:详细的设计方案,相关的数据结构、算法描述,可采用伪代码等形式化描述
	% 实现:修改哪些类、如何修改、为什么修改等;
	% 测试:测试用例,测试结果及结果分析,测试运行界面等;
	% 调试:调试方法,遇到的问题及解决方案等;
	% 结论与展望:完成的主要工作、收获、进一步的工作,建议、体会、心得等;

\section{Experiment 1: Linear Regression}
\subsection{2D Linear Regression}

首先读入数据到 matlab:

\begin{lstlisting}[language=matlab,title={matlab load}]
x = load('ex1x.dat');
y = load('ex1y.dat');
\end{lstlisting}

使用 plot 函数可以进行绘制:

\begin{lstlisting}[language=matlab,title={matlab figure}]
figure % open a new figure window
plot(x,y,'o');
ylabel('Height in meters')
xlabel('Age in years')
\end{lstlisting}

得到的图像为:

\begin{figure}[H]
    \centering
    \includesvg[width=12cm]{imgs/1.svg}
\end{figure} 

将 $x$ 增加偏置项,这样就可以进行线性回归:

\begin{lstlisting}[language=matlab,title={matlab regression init}]
m=length(y); % store the number of training examples
x=[ones(m,1),x]; % Add a column of ones to x
\end{lstlisting}

线性回归的计算模型为:

\[
h_{\theta}(x)=\theta^{T} x=\sum_{i=0}^{n} \theta_{i} x_{i}
\]

线性回归梯度下降的公式为:

\[
\theta_{j}:=\theta_{j}-\alpha \frac{1}{m} \sum_{i=1}^{m}\left(h_{\theta}\left(x^{(i)}\right)-y^{(i)}\right) x_{j}^{(i)}
\]

\begin{lstlisting}[language=matlab,title={Linear Regression}]
alpha=0.07;
iterations=1500;
theta=zeros(2,1);

for iter=1:iterations
    h=x*theta;
    error=h-y;
    theta=theta-(alpha/m)*(x'*error);
end

disp(theta);
\end{lstlisting}

输出 0.7502 与 0.0639,也就是 $\theta_1=0.7502, \theta_2=0.0639$,可以在原图像基础上绘制回归直线:

\begin{lstlisting}[language=matlab,title={figure regression}]
hold on
plot(x(:,2),x*theta,'-')
legend('Training data','Linear regression')
\end{lstlisting}

得到图像:

\begin{figure}[H]
    \centering
    \includesvg[width=12cm]{imgs/1-2.svg}
\end{figure} 

计算 3.5 岁与 7 岁同学的身高,直接代入得:

\begin{lstlisting}[language=matlab,title={linear predict}]
disp([1,3.5]*theta)
disp([1,7]*theta)
\end{lstlisting}

得到 0.9737 与 1.1973。

% \begin{figure}[H]
%     \centering
%     \includegraphics[width = 15cm]{imgs/hw1_3.jpg}
% \end{figure} 

% \begin{figure}[H]
%     \centering
%     \includegraphics[width = 15cm]{imgs/hw1_4.jpg}
% \end{figure} 

\subsection{Understanding $J(\theta)$}

由 $J(\theta )=\frac{1}{2m}\sum _{i=1}^n\left (h_\theta (x_i)-y_i\right ) ^2$,不难写出代码:

\begin{lstlisting}[language=matlab,title={calculate $J(\theta)$}]
theta1_vals=linspace(-10,10,100);
theta2_vals=linspace(-10,10,100);
[J,theta1,theta2]=meshgrid(theta1_vals,theta2_vals);
J_vals=zeros(size(theta1));

for i=1:size(theta1,1)
    for j=1:size(theta1,2)
        theta_temp=[theta1(i,j);theta2(i,j)];
        h=x*theta_temp;
        J_vals(i,j)=(1/(2*m))*sum((h-y).^2);
    end
end
\end{lstlisting}


\begin{figure}[H]
    \centering
    \includesvg[width=12cm]{imgs/1-3.svg}
\end{figure} 

不太直观,实际上还可以绘制梯度下降中迭代的“路径”:

\begin{figure}[H]
    \centering
    \includesvg[width=12cm]{imgs/1-4.svg}
\end{figure} 

\begin{figure}[H]
    \centering
    \includesvg[width=12cm]{imgs/1-5.svg}
\end{figure} 

\subsection{Programming}

绘制路径部分的完整代码如下:

\begin{lstlisting}[language=matlab,title={figure surface \& contour plot}]
x=load('ex1x.dat');
y=load('ex1y.dat');
m=length(y);
x=[ones(m,1),x];
theta1_vals=linspace(-10,10,50);
theta2_vals=linspace(-10,10,50);
[theta1,theta2]=meshgrid(theta1_vals,theta2_vals);
J_vals=zeros(size(theta1));

for i=1:size(theta1,1)
    for j=1:size(theta1,2)
        theta_temp=[theta1(i,j);theta2(i,j)];
        h=x*theta_temp;
        J_vals(i,j)=(1/(2*m))*sum((h-y).^2);
    end
end

alpha=0.07;
iterations=1000;
theta=[6;7.5];
J_history=zeros(iterations,1);
theta_history=zeros(iterations,2);

for iter=1:iterations
    h=x*theta;
    error=h-y;
    theta=theta-(alpha/m)*(x'*error);
    J_history(iter)=(1/(2*m))*sum((h-y).^2);
    theta_history(iter,:)=theta';
end

figure;
surf(theta1,theta2,J_vals);
hold on;

view(30,30);
shading interp;
colorbar;

final_cost=(1/(2*m))*sum((x*theta-y).^2);
plot3(theta(1),theta(2),final_cost,'r*','MarkerSize',10);
xlabel('\theta_1');
ylabel('\theta_2');
zlabel('J(\theta)');
title('Surface plot of J(\theta) for Linear Regression');

for i=1:10:iterations
    if i+10<=iterations
        plot3(theta_history(i,1),theta_history(i,2),...
              (1/(2*m))*sum((x*theta_history(i,:)'-y).^2),'bo','MarkerSize',5);
        quiver3(theta_history(i,1),theta_history(i,2),...
                 (1/(2*m))*sum((x*theta_history(i,:)'-y).^2),...
                 theta_history(i+10,1)-theta_history(i,1),...
                 theta_history(i+10,2)-theta_history(i,2),...
                 0,'k','LineWidth',2,'AutoScale','off');
    end
end

gradient=(1/m)*(x'*(h-y));
quiver3(theta(1),theta(2),final_cost,gradient(1),gradient(2),0,'k','LineWidth',2,'AutoScale','off');

text(theta(1),theta(2),final_cost,sprintf('  (%.2f,%.2f)',theta(1),theta(2)),'Color','red');

hold off;

print('surface_plot.svg','-dsvg');

figure;
contour(theta1,theta2,J_vals,50);
hold on;

plot(theta(1),theta(2),'r*','MarkerSize',10);
xlabel('\theta_1');
ylabel('\theta_2');
title('Contour plot of J(\theta) for Linear Regression');
colorbar;

num_arrows=10;
for i=1:num_arrows:iterations
    if i+num_arrows<=iterations
        quiver(theta_history(i,1),theta_history(i,2),...
               theta_history(i+num_arrows,1)-theta_history(i,1),...
               theta_history(i+num_arrows,2)-theta_history(i,2),...
               'k','LineWidth',2,'AutoScale','off');
    end
end

hold off;
\end{lstlisting}

\newpage
	% 代码分析:模块功能、涉及到的类、类关系、数据结构及关键代码等;
	% 任务要求,设计任务要求;
	% 设计:详细的设计方案,相关的数据结构、算法描述,可采用伪代码等形式化描述
	% 实现:修改哪些类、如何修改、为什么修改等;
	% 测试:测试用例,测试结果及结果分析,测试运行界面等;
	% 调试:调试方法,遇到的问题及解决方案等;
	% 结论与展望:完成的主要工作、收获、进一步的工作,建议、体会、心得等;

\section{Experiment 2: Multivariate Linear Regression}

\subsection{预处理数据}

首先读取数据:

\begin{lstlisting}[language=matlab,title={读入 ex2?.dat}]
x = load('ex2x.dat');
y = load('ex2y.dat');
m = length(y);
x = [ones(m, 1), x];
\end{lstlisting}

对数据进行特征缩放,也就是进行标准化,公式为:

\[
x_{i,\text{scaled}}=\frac{x_i-\mu_i}{\sigma_i}
\]

其中 $\mu_i$ 为均值、$\sigma_i$ 为标准差,在 matlab 中有函数 mean() 与 std() 可以计算,如此可让不同数量级的特征位于统一尺度下、加速收敛。

\begin{lstlisting}[language=matlab,title={标准化 x}]
sigma = std(x);
mu = mean(x);
x(:, 2) = (x(:, 2) - mu(2)) / sigma(2);
x(:, 3) = (x(:, 3) - mu(3)) / sigma(3);
\end{lstlisting}

\subsection{$ J(\mathbf \theta )$ 的计算}

$J(\mathbf \theta ) $ 为代价函数,表示迭代的代价。计算公式如下:

\begin{align}
    h_{\mathbf \theta }(\mathbf x)&=\mathbf \theta ^T\mathbf x=\sum_{i = 0}^{n} \theta _ix_i\\
    \theta _j &=\theta _j-\alpha \frac{1}{m}\sum_{i = 1}^{m}  (h_{\theta } (x^{(i)})-y^{(i)})x_j ^i\\
    J(\mathbf \theta )&=\frac{1}{2m} (\mathbf X\mathbf \theta -\mathbf y)^T(\mathbf X\mathbf \theta -\mathbf y)
\end{align}

于是不难写出代码(加入 J\_history 记录 $J(\mathbf \theta ) $ 的值方便追踪过程):

\begin{lstlisting}[language=matlab,title={gradientDescent.m}]
function [theta, J_history] = gradientDescent(x, y, theta, alpha, iterations)
    m = length(y);
    J_history = zeros(iterations, 1);
    
    for i = 1:iterations
        error = x * theta - y;
        J_history(i) = (error' * error) / (2 * m);
        
        gradient = (x' * error) / m;
        theta = theta - alpha * gradient;
    end
end
\end{lstlisting}

\subsection{选择学习率}

首先设置学习参数:

\begin{lstlisting}[language=matlab,title={梯度下降参数}]
iterations = 50;
theta_init = zeros(size(x, 2), 1);
\end{lstlisting}

可用 gradientDescent 测试不同学习率下收敛速率:

\begin{lstlisting}[language=matlab,title={对比不同学习率}]
[theta1, J1] = gradientDescent(x, y, theta_init, 1, iterations);
[theta2, J2] = gradientDescent(x, y, theta_init, 0.2, iterations);
[theta3, J3] = gradientDescent(x, y, theta_init, 0.04, iterations);
\end{lstlisting}

绘图,得:

\begin{figure}[H]
    \centering
    \includesvg[width=10cm]{imgs/gradientDescent.svg}
\end{figure} 

\begin{lstlisting}[language=matlab,title={画图}]
figure;
plot(0:iterations-1, J1, 'r-', 'DisplayName', 'alpha = 1');
hold on;
plot(0:iterations-1, J2, 'b-', 'DisplayName', 'alpha = 0.2');
plot(0:iterations-1, J3, 'g-', 'DisplayName', 'alpha = 0.04');
legend;
xlabel('Iterations');
ylabel('Cost J');
hold off;
\end{lstlisting}

\subsection{questions}

\begin{lstlisting}[language=c++,title={5.1 - 问题描述}]
Observe the changes in the cost function happens as the learning rate changes. What happens when the learning rate is too small? Too large?
\end{lstlisting}

当 $\alpha$ 太小时,梯度下降收敛速度很慢,参数更新幅度很小,需要更多迭代才能接近最优解;当 $\alpha$ 太大时,梯度下降可能会越过最优解,导致 $J(\mathbf{\theta})$ 震荡,甚至无法收敛。

\begin{lstlisting}[language=c++,title={5.2 - 问题描述}]
Using the best learning rate that you found, run gradient descent until convergence to find
(a) The final values of theta
(b) The predicted price of a house with 1650 square feet and 3 bedrooms. Don’t forget to scale your features when you make this prediction!
\end{lstlisting}

\begin{enumerate}
    \item $J(\mathbf \theta) = 2.043\times 10^9, \mathbf \theta =[3.404\times 10^5,1.106\times 10^5,-6.625\times 10^3]$;
    \item 预测值为 $2.93\times 10^5$。
\end{enumerate}

\begin{lstlisting}[language=c++,title={6.1 - 问题描述}]
In your program, use the formula above to calculate theta. Remember that while you don’t need to scale your features, you still need to add an intercept term.
\end{lstlisting}

结果为$\theta =[8.960\times 10^4,1.392\times 10^2,-8.738\times 10^3]$。

\begin{lstlisting}[language=c++,title={6.2 - 问题描述}]
Once you have found theta from this method, use it to make a price prediction for a 1650-square-foot house with 3 bedrooms. Did you get the same price that you found through gradient descent?
\end{lstlisting}

结果为 $2.931\times 10^5$。

\subsection{完整代码}

gradientDescent.m 在上文中已给出

\begin{lstlisting}[language=matlab,title={完整代码}]
clear;
x = load('ex2x.dat');
y = load('ex2y.dat');
m = length(y);
x = [ones(m, 1), x];

% 特征缩放
sigma = std(x);
mu = mean(x);
x(:, 2) = (x(:, 2) - mu(2)) / sigma(2);
x(:, 3) = (x(:, 3) - mu(3)) / sigma(3);

% 梯度下降参数
iterations = 50;
theta_init = zeros(size(x, 2), 1);

% 不同学习率下的梯度下降
[theta1, J1] = gradientDescent(x, y, theta_init, 1, iterations);
[theta2, J2] = gradientDescent(x, y, theta_init, 0.2, iterations);
[theta3, J3] = gradientDescent(x, y, theta_init, 0.04, iterations);

% 绘图
figure;
plot(0:iterations-1, J1, 'r-', 'DisplayName', 'alpha = 1');
hold on;
plot(0:iterations-1, J2, 'b-', 'DisplayName', 'alpha = 0.2');
plot(0:iterations-1, J3, 'g-', 'DisplayName', 'alpha = 0.04');
legend;
xlabel('Iterations');
ylabel('Cost J');
hold off;
\end{lstlisting}
\newpage
	% 代码分析:模块功能、涉及到的类、类关系、数据结构及关键代码等;
	% 任务要求,设计任务要求;
	% 设计:详细的设计方案,相关的数据结构、算法描述,可采用伪代码等形式化描述
	% 实现:修改哪些类、如何修改、为什么修改等;
	% 测试:测试用例,测试结果及结果分析,测试运行界面等;
	% 调试:调试方法,遇到的问题及解决方案等;
	% 结论与展望:完成的主要工作、收获、进一步的工作,建议、体会、心得等;

\section{Experiment 3: Logistic Regression and Newton's Method}

\subsection{预处理数据}

首先读取数据:

\begin{lstlisting}[language=matlab,title={读入 ex4?.dat}]
x = load('ex4x.dat');
y = load('ex4y.dat');

[m, n] = size(x);
x = [ones(m, 1), x];
\end{lstlisting}

\subsection{输入数据可视化}

\begin{lstlisting}[language=matlab,title={绘制图像}]
pos = find(y == 1);
neg = find(y == 0);

figure; hold on;
plot(x(pos, 2), x(pos, 3), 'k+', 'LineWidth', 2, 'MarkerSize', 7);
plot(x(neg, 2), x(neg, 3), 'ko', 'MarkerFaceColor', 'y', 'MarkerSize', 7);
xlabel('考试1成绩');
ylabel('考试2成绩');
legend('已录取', '未录取');
hold off;
\end{lstlisting}

\begin{figure}[H]
    \centering
    \includesvg[width=10cm]{imgs/3-1.svg}
\end{figure} 

\subsection{牛顿法}

这里我将 g 写成了一个函数:

\begin{lstlisting}[language=matlab,title={sigmoid.m}]
function g = sigmoid(z)
    g = 1.0 ./ (1.0 + exp(-z));
end
\end{lstlisting}

可以对着写出 $J(\theta)$ 的函数:

\[
J\left( \theta \right)  = \frac{1}{m}\mathop{\sum }\limits_{{i = 1}}^{m}\left\lbrack  {-{y}^{\left( i\right) }\log \left( {{h}_{\theta }\left( {x}^{\left( i\right) }\right) }\right)  - \left( {1 - {y}^{\left( i\right) }}\right) \log \left( {1 - {h}_{\theta }\left( {x}^{\left( i\right) }\right) }\right) }\right\rbrack
\]

\begin{lstlisting}[language=matlab,title={costFunction.m}]
function [J, grad] = costFunction(theta, x, y)
    m = length(y);
    h = sigmoid(x * theta);
    
    J = (1/m) * sum(-y .* log(h) - (1 - y) .* log(1 - h));
    grad = (1/m) * (x' * (h - y));
end

\end{lstlisting}

Hessian 矩阵为:

\[
{\nabla }_{\theta }J = \frac{1}{m}\mathop{\sum }\limits_{{i = 1}}^{m}\left( {{h}_{\theta }\left( {x}^{\left( i\right) }\right)  - {y}^{\left( i\right) }}\right) {x}^{\left( i\right) }
\]

故可以在牛顿法写出:

\[
{\theta }^{\left( t + 1\right) } = {\theta }^{\left( t\right) } - {H}^{-1}{\nabla }_{\theta }J
\]

\begin{lstlisting}[language=matlab,title={newtonsMethod.m}]
function [theta, J_history] = newtonsMethod(x, y, theta, max_iters)
    m = length(y);
    J_history = zeros(max_iters, 1);

    for iter = 1:max_iters
        h = sigmoid(x * theta);
        grad = (1/m) * (x' * (h - y));
        H = (1/m) * (x' * diag(h .* (1 - h)) * x);
        theta = theta - H\grad;

        J_history(iter) = (1/m) * sum(-y .* log(h) - (1 - y) .* log(1 - h));
    end
end
\end{lstlisting}

然后可以画出决策边界:

\begin{lstlisting}[language=matlab,title={plotDecisionBoundary.m}]
function plotDecisionBoundary(theta, x, y)
    pos = find(y == 1);
    neg = find(y == 0);
    
    figure; hold on;
    plot(x(pos, 2), x(pos, 3), 'k+', 'LineWidth', 2, 'MarkerSize', 7);
    plot(x(neg, 2), x(neg, 3), 'ko', 'MarkerFaceColor', 'y', 'MarkerSize', 7);
    
    plot_x = [min(x(:,2))-2, max(x(:,2))+2];
    plot_y = (-1./theta(3)) .* (theta(2).*plot_x + theta(1));
    
    plot(plot_x, plot_y, 'b-', 'LineWidth', 2);
    
    xlabel('考试1成绩');
    ylabel('考试2成绩');
    legend('已录取', '未录取', '决策边界');
    hold off;
end
\end{lstlisting}

\begin{figure}[H]
    \centering
    \includesvg[width=10cm]{imgs/3-2.svg}
\end{figure} 

\subsection{questions}

\begin{lstlisting}[language=c++,title={4.1 - 问题描述}]
What values of theta did you get? How many iterations were required for convergence?(你得到的 theta 值是多少?需要多少次迭代才能收敛?)
\end{lstlisting}

\begin{lstlisting}[language=matlab,title={4.1 - 求 theta 值与观察收敛速度}]
figure;
plot(1:max_iters, J_history(1:max_iters), '-b', 'LineWidth', 2);
xlabel('迭代次数');
ylabel('成本函数 J(theta)');
title('成本函数随迭代次数的变化');

fprintf('优化后的 theta 值为:\n');
disp(theta);
fprintf('收敛所需的迭代次数:%d\n', max_iters);
\end{lstlisting}

得到值为:

\begin{lstlisting}[language=matlab,title={4.1 - $\theta$ 值}]
优化后的 theta 值为:
  -16.3787
    0.1483
    0.1589
\end{lstlisting}

以及对应的函数图像:

\begin{figure}[H]
    \centering
    \includesvg[width=10cm]{imgs/3-3.svg}
\end{figure} 

可以看出,在 5 次迭代时就已收敛。

\begin{lstlisting}[language=c++,title={4.2 - 问题描述}]
What is the probability that a student with a score of 20 on Exam 1 and a score of 80 on Exam 2 will not be admitted?(考试 1 的成绩为 20 分,考试 2 的成绩为 80 分的学生不被录取的概率是多少?)
\end{lstlisting}

\begin{lstlisting}[language=matlab,title={4.2 - 预测概率}]
student_scores = [1, 20, 80];
prob = sigmoid(student_scores * theta);

fprintf('不被录取的概率: %f\n', prob);
\end{lstlisting}

得到结果:

\begin{lstlisting}[language=matlab,title={预测概率}]
不被录取的概率: 0.331978
\end{lstlisting}

\subsection{完整代码}

\begin{lstlisting}[language=matlab,title={完整代码}]
clear; clc;

x = load('ex4x.dat');
y = load('ex4y.dat');

[m, n] = size(x);
x = [ones(m, 1), x];

pos = find(y == 1);
neg = find(y == 0);

figure; hold on;
plot(x(pos, 2), x(pos, 3), 'k+', 'LineWidth', 2, 'MarkerSize', 7);
plot(x(neg, 2), x(neg, 3), 'ko', 'MarkerFaceColor', 'y', 'MarkerSize', 7);
xlabel('考试1成绩');
ylabel('考试2成绩');
legend('已录取', '未录取');
hold off;

initial_theta = zeros(n + 1, 1);
max_iters = 15;
[theta, J_history] = newtonsMethod(x, y, initial_theta, max_iters);

% 绘制决策边界
plotDecisionBoundary(theta, x, y);

figure;
plot(1:max_iters, J_history(1:max_iters), '-b', 'LineWidth', 2);
xlabel('迭代次数');
ylabel('成本函数 J(theta)');
title('成本函数随迭代次数的变化');

fprintf('优化后的 theta 值为:\n');
disp(theta);
fprintf('收敛所需的迭代次数:%d\n', max_iters);

student_scores = [1, 20, 80];
prob = sigmoid(student_scores * theta);

fprintf('不被录取的概率: %f\n', prob);
\end{lstlisting}
\newpage
	% 代码分析:模块功能、涉及到的类、类关系、数据结构及关键代码等;
	% 任务要求,设计任务要求;
	% 设计:详细的设计方案,相关的数据结构、算法描述,可采用伪代码等形式化描述
	% 实现:修改哪些类、如何修改、为什么修改等;
	% 测试:测试用例,测试结果及结果分析,测试运行界面等;
	% 调试:调试方法,遇到的问题及解决方案等;
	% 结论与展望:完成的主要工作、收获、进一步的工作,建议、体会、心得等;

\section{Experiment 4.1 Regularized Linear Regression}

\subsection{预处理数据}

首先读取数据:

\begin{lstlisting}[language=matlab,title={读入 ex5Lin?.dat}]
clear;
% Load linear regression data
x = load('ex5Linx.dat');
y = load('ex5Liny.dat');
m = length(y); % Number of training examples
\end{lstlisting}

\subsection{数据可视化}

\begin{lstlisting}[language=matlab,title={数据可视化}]
% Visualize the data
figure;
plot(x, y, 'rx', 'MarkerSize', 10); % Scatter plot
xlabel('x');
ylabel('y');
title('Training Data');
\end{lstlisting}

\begin{figure}[H]
    \centering
    \includesvg[width=10cm]{imgs/4-1-1.svg}
\end{figure} 

\subsection{线性回归}

使用五阶多项式:

\[
{h}_{\theta }\left( x\right)  = {\theta }_{0} + {\theta }_{1}x + {\theta }_{2}{x}^{2} + {\theta }_{3}{x}^{3} + {\theta }_{4}{x}^{4} + {\theta }_{5}{x}^{5}
\]

由于我们正在将 5 阶多项式拟合到只有 7 个点的数据集,因此很可能会发生过度拟合。为了防止这种情况,我们将在模型中使用正则化:也就是最小化正则化 Cost:

\[
J\left( \theta \right)  = \frac{1}{2m}\left\lbrack  {\mathop{\sum }\limits_{{i = 1}}^{m}{\left( {h}_{\theta }\left( {x}^{\left( i\right) }\right)  - {y}^{\left( i\right) }\right) }^{2} + \lambda \mathop{\sum }\limits_{{j = 1}}^{n}{\theta }_{j}^{2}}\right\rbrack
\]

在课上学习了正则化线性回归的正则方程解:

\[
\theta  = {\left( {X}^{T}X + \lambda \left\lbrack  \begin{array}{llll} 0 & & & \\   & 1 & & \\   & &  \ddots  & \\   & & & 1 \end{array}\right\rbrack  \right) }^{-1}{X}^{T}\overrightarrow{y}
\]

于是不难写出代码:

\begin{lstlisting}[language=matlab,title={RegularizedLinearRegression.m}]
clear;
% Load linear regression data
x = load('ex5Linx.dat');
y = load('ex5Liny.dat');
m = length(y); % Number of training examples

% Visualize the data
figure;
plot(x, y, 'rx', 'MarkerSize', 10); % Scatter plot
xlabel('x');
ylabel('y');
title('Training Data');

X = [ones(m, 1), x, x.^2, x.^3, x.^4, x.^5];

lambda_vals = [0, 1, 10];
theta_values = zeros(6, length(lambda_vals));

for i = 1:length(lambda_vals)
    lambda = lambda_vals(i);
    L = lambda * diag([0; ones(5, 1)]); % Regularization matrix
    theta = (X' * X + L) \ (X' * y);    % Normal equation with regularization
    theta_values(:, i) = theta;
    
    % Plot the polynomial fit
    x_plot = linspace(min(x), max(x), 100)';
    X_plot = [ones(size(x_plot, 1), 1), x_plot, x_plot.^2, x_plot.^3, x_plot.^4, x_plot.^5];
    y_plot = X_plot * theta;
    
    figure;
    plot(x, y, 'rx', 'MarkerSize', 10);
    hold on;
    plot(x_plot, y_plot, '-b', 'LineWidth', 2);
    xlabel('x');
    ylabel('y');
    title(['Polynomial Fit for \lambda = ', num2str(lambda)]);
end

for i = 1:length(lambda_vals)
    norm_theta = norm(theta_values(:, i));
    fprintf('L2 norm of theta for lambda = %d: %.4f\n', lambda_vals(i), norm_theta);
end
\end{lstlisting}

这里对 $\lambda=\{0,1,10\}$ 进行了计算,可以得到拟合的图像为:

\begin{figure}[H]
    \centering
    \includesvg[width=5.5cm]{imgs/4-1-2.svg}
    \includesvg[width=5.5cm]{imgs/4-1-3.svg}
    \includesvg[width=5.5cm]{imgs/4-1-4.svg}
\end{figure} 

\subsection{question}

\begin{lstlisting}[language=matlab,title={问题描述}]
From looking at these graphs, what conclusions can you make about how the regularization parameter lambda affects your model?
\end{lstlisting}

\begin{itemize}
    \item 较小的 \( \lambda \) 值( \( \lambda = 0 \)):本例中模型过度拟合,方差较大。
    \item 适中的 \( \lambda \) 值( \( \lambda = 1 \)):本例中看起来还可以
    \item 较大的 \( \lambda \) 值( \( \lambda = 10 \)):本例中欠拟合,模型的偏差较大,无法有效拟合数据。
\end{itemize}

\section{Experiment 4.2 Regularized Logistic Regression}

\subsection{预处理数据}

首先读取数据:

\begin{lstlisting}[language=matlab,title={读入 ex5Log?.dat}]
clear;
data = load('ex5Logx.dat');
X = data(:, 1:2);
y = load('ex5Logy.dat');
\end{lstlisting}

\subsection{数据可视化}

\begin{lstlisting}[language=matlab,title={数据可视化}]
figure;
pos = find(y == 1);
neg = find(y == 0);

plot(X(pos, 1), X(pos, 2), 'k+', 'LineWidth', 2, 'MarkerSize', 7);
hold on;
plot(X(neg, 1), X(neg, 2), 'ko', 'MarkerFaceColor', 'y', 'MarkerSize', 7);
xlabel('Feature u');
ylabel('Feature v');
title('Logistic Regression Data');
legend('Positive', 'Negative');
hold off;
\end{lstlisting}

\begin{figure}[H]
    \centering
    \includesvg[width=10cm]{imgs/4-2-1.svg}
\end{figure} 

\subsection{特征映射}

给出了 map\_feature.m,直接调用来进行多项式特征映射:

\begin{lstlisting}[language=matlab,title={特征映射}]
mapped_X = map_feature(X(:, 1), X(:, 2));
\end{lstlisting}

\subsection{牛顿法求解}

regularized logistic regression 的 cost 函数为:

\[
J\left( \theta \right)  =  - \frac{1}{m}\mathop{\sum }\limits_{{i = 1}}^{m}\left\lbrack  {{y}^{\left( i\right) }\log \left( {{h}_{\theta }\left( {x}^{\left( i\right) }\right) }\right)  + \left( {1 - {y}^{\left( i\right) }}\right) \log \left( {1 - {h}_{\theta }\left( {x}^{\left( i\right) }\right) }\right) }\right\rbrack   + \frac{\lambda }{2m}\mathop{\sum }\limits_{{j = 1}}^{n}{\theta }_{j}^{2}
\]

梯度 ${\nabla }_{\theta }J$ 为:

\[
{\nabla }_{\theta }J = \left\lbrack  \begin{matrix} \frac{1}{m}\mathop{\sum }\limits_{{i = 1}}^{m}\left( {{h}_{\theta }\left( {x}^{\left( i\right) }\right)  - {y}^{\left( i\right) }}\right) {x}_{0}^{\left( i\right) } \\  \frac{1}{m}\mathop{\sum }\limits_{{i = 1}}^{m}\left( {{h}_{\theta }\left( {x}^{\left( i\right) }\right)  - {y}^{\left( i\right) }}\right) {x}_{1}^{\left( i\right) } + \frac{\lambda }{m}{\theta }_{1} \\ \vdots \\  \frac{1}{m}\mathop{\sum }\limits_{{i = 1}}^{m}\left( {{h}_{\theta }\left( {x}^{\left( i\right) }\right)  - {y}^{\left( i\right) }}\right) {x}_{1}^{\left( i\right) } + \frac{\lambda }{m}{\theta }_{n} \end{matrix}\right\rbrack
\]

Hessian 矩阵为:

\[
H = \frac{1}{m}\left\lbrack  {\mathop{\sum }\limits_{{i = 1}}^{m}{h}_{\theta }\left( {x}^{\left( i\right) }\right) \left( {1 - {h}_{\theta }\left( {x}^{\left( i\right) }\right) }\right) {x}^{\left( i\right) }{\left( {x}^{\left( i\right) }\right) }^{T}}\right\rbrack   + \frac{\lambda }{m}\left\lbrack  \begin{array}{llll} 0 & & & \\   & 1 & & \\   & &  \ddots  & \\   & & & 1 \end{array}\right\rbrack
\]

\begin{lstlisting}[language=matlab,title={RegularizedLogisticRegression.m}]
initial_theta = zeros(size(mapped_X, 2), 1);

lambda_vals = [0, 1, 10];

for i = 1:length(lambda_vals)
    lambda = lambda_vals(i);
    
    [theta, J_history] = newtons_method(mapped_X, y, initial_theta, lambda);
    
    figure;
    plot_decision_boundary(theta, X, y);
    title(['Decision Boundary with \lambda = ', num2str(lambda)]);
end
\end{lstlisting}

其中 newtons\_method 和 plot\_decision\_boundary 来自我在 ex4 中的代码,由于几乎一样稍微改改就可以使用,于是代码放在结尾了。

这里计算了 $\lambda=\{0,1,10\}$ 的数据,可以画出:

\begin{figure}[H]
    \centering
    \includesvg[width=5.5cm]{imgs/4-2-2.svg}
    \includesvg[width=5.5cm]{imgs/4-2-3.svg}
    \includesvg[width=5.5cm]{imgs/4-2-4.svg}
\end{figure} 

\subsection{question}

\begin{lstlisting}[language=matlab,title={问题描述}]
How does lambda affect the results?
\end{lstlisting}

与前一个实验其实是类似的:

\begin{itemize}
    \item 较小的 \( \lambda \) 值(如 \( \lambda = 0 \)):本例中导致模型过度拟合,决策边界过于复杂。
    \item 适中的 \( \lambda \) 值(如 \( \lambda = 1 \)):本例中看起来还好。
    \item 较大的 \( \lambda \) 值(如 \( \lambda = 10 \)):本例中导致欠拟合,决策边界过于简单,无法有效区分数据。
\end{itemize}

\subsection{完整代码}

\begin{lstlisting}[language=matlab,title={newtons\_method.m}]
function [theta, J_history] = newtons_method(X, y, initial_theta, lambda)
    % X: feature matrix after mapping
    % y: labels
    % initial_theta: initial parameters
    % lambda: regularization parameter

    % Initialize useful values
    [m, n] = size(X); % m = number of training examples, n = number of features
    theta = initial_theta;
    J_history = []; % To store the cost function values

    max_iter = 50;  % Maximum number of iterations
    tolerance = 1e-6;  % Tolerance for convergence

    for iter = 1:max_iter
        % Compute the hypothesis h_theta(x)
        z = X * theta;
        h = sigmoid(z);

        % Compute the cost function J with regularization (excluding theta(1))
        J = -(1/m) * (y' * log(h) + (1 - y)' * log(1 - h)) + ...
            (lambda/(2*m)) * sum(theta(2:end).^2);

        % Gradient (with regularization)
        grad = (1/m) * (X' * (h - y)) + [0; (lambda/m) * theta(2:end)];

        % Hessian (with regularization)
        H = (1/m) * (X' * diag(h .* (1 - h)) * X) + lambda/m * diag([0; ones(n-1, 1)]);

        % Update theta using Newton's method
        delta_theta = H \ grad;
        theta = theta - delta_theta;

        % Store cost function value for each iteration
        J_history = [J_history; J];

        % Check for convergence
        if norm(delta_theta) < tolerance
            fprintf('Converged after %d iterations\n', iter);
            break;
        end
    end
end

% Sigmoid function used in logistic regression
function g = sigmoid(z)
    g = 1 ./ (1 + exp(-z));
end
\end{lstlisting}

\begin{lstlisting}[language=matlab,title={plot\_decision\_boundary.m}]
function plot_decision_boundary(theta, X, y)
    u_vals = linspace(min(X(:,1))-0.1, max(X(:,1))+0.1, 200);
    v_vals = linspace(min(X(:,2))-0.1, max(X(:,2))+0.1, 200);
    
    z = zeros(length(u_vals), length(v_vals));

    for i = 1:length(u_vals)
        for j = 1:length(v_vals)
            z(i, j) = map_feature(u_vals(i), v_vals(j)) * theta;
        end
    end

    contour(u_vals, v_vals, z', [0, 0], 'LineWidth', 2);
    hold on;
    
    pos = find(y == 1);
    neg = find(y == 0);
    plot(X(pos, 1), X(pos, 2), 'k+', 'LineWidth', 2, 'MarkerSize', 7);
    plot(X(neg, 1), X(neg, 2), 'ko', 'MarkerFaceColor', 'y', 'MarkerSize', 7);
    
    xlabel('Feature u');
    ylabel('Feature v');
    title('Decision Boundary');
    hold off;
end
\end{lstlisting}
\newpage
	% 代码分析:模块功能、涉及到的类、类关系、数据结构及关键代码等;
	% 任务要求,设计任务要求;
	% 设计:详细的设计方案,相关的数据结构、算法描述,可采用伪代码等形式化描述
	% 实现:修改哪些类、如何修改、为什么修改等;
	% 测试:测试用例,测试结果及结果分析,测试运行界面等;
	% 调试:调试方法,遇到的问题及解决方案等;
	% 结论与展望:完成的主要工作、收获、进一步的工作,建议、体会、心得等;

\section{Experiment 5.1 LDA for 2 Classes}

\subsection{预处理数据}

首先读取数据:

\begin{lstlisting}[language=matlab,title={读入 ex3?.dat}]
blueData = load('../exp5/ex3blue.dat');
redData = load('../exp5/ex3red.dat');
\end{lstlisting}

\subsection{数据可视化}

\begin{lstlisting}[language=matlab,title={plotData.m}]
function plotData(blueData, redData)
    figure;
    scatter(blueData(:, 1), blueData(:, 2), 'b', 'filled');
    hold on;
    scatter(redData(:, 1), redData(:, 2), 'r', 'filled');
    xlabel('X');
    ylabel('Y');
    title('Scatter Plot with Projection Line');
    legend({'Blue Point', 'Red Point'}, 'Location', 'northeast');
    axis equal;
end
\end{lstlisting}

\begin{figure}[H]
    \centering
    \includesvg[width=10cm]{imgs/5-1-1.svg}
\end{figure} 

\subsection{Linear Discriminant Analysis-Two Classes}

首先求出均值向量:

\[
{\mathbf{\mu }}_{i} = \frac{1}{{n}_{i}}\mathop{\sum }\limits_{{x \in  {C}_{i}}}\mathbf{x}\quad{\widetilde{\mathbf{\mu }}}_{i} = \frac{1}{{n}_{i}}\mathop{\sum }\limits_{{y \in  {C}_{i}}}y = \frac{1}{{n}_{i}}\mathop{\sum }\limits_{{x \in  {C}_{i}}}{\mathbf{\theta }}^{T}\mathbf{x} = {\mathbf{\theta }}^{T}{\mathbf{\mu }}_{i}
\]

\begin{lstlisting}[language=matlab,title={computeMeans.m}]
function [meanBlue, meanRed] = computeMeans(blueData, redData)
    meanBlue = mean(blueData);
    meanRed = mean(redData);
end
\end{lstlisting}

需要最大化两类之间的间距:

\[
{J}_{1}\left( \mathbf{\theta }\right)  = {\left( {\widetilde{\mu }}_{1} - {\widetilde{\mu }}_{2}\right) }^{2} = {\left( {\mathbf{\theta }}^{T}{\mathbf{\mu }}_{1} - {\mathbf{\theta }}^{T}{\mathbf{\mu }}_{2}\right) }^{2} = {\mathbf{\theta }}^{T}\left( {{\mathbf{\mu }}_{1} - {\mathbf{\mu }}_{2}}\right) {\left( {\mathbf{\mu }}_{1} - {\mathbf{\mu }}_{2}\right) }^{T}\mathbf{\theta } = {\mathbf{\theta }}^{T}{\mathbf{S}}_{b}\mathbf{\theta }
\]

其中 $\mathbf{S}_b$ 称为类间散度矩阵 ${\mathbf{S}}_{b} = \left( {{\mathbf{\mu }}_{1} - {\mathbf{\mu }}_{2}}\right) {\left( {\mathbf{\mu }}_{1} - {\mathbf{\mu }}_{2}\right) }^{T}$。

还需要最小化类内的间距:

\[
{J}_{2}\left( \mathbf{\theta }\right)  = \mathop{\sum }\limits_{{y \in  {C}_{1}}}{\left( y - {\widetilde{\mu }}_{1}\right) }^{2} + \mathop{\sum }\limits_{{y \in  {C}_{2}}}{\left( y - {\widetilde{\mu }}_{2}\right) }^{2} = {\mathbf{\theta }}^{T}{\mathbf{S}}_{w}\mathbf{\theta }
\]

其中 $\mathbf{S}_w$ 称为类内散度矩阵 ${S}_{w} = \mathop{\sum }\limits_{{x \in  {C}_{1}}}\left( {x - {\mu }_{1}}\right) {\left( x - {\mu }_{1}\right) }^{T} + \mathop{\sum }\limits_{{x \in  {C}_{2}}}\left( {x - {\mu }_{2}}\right) {\left( x - {\mu }_{2}\right) }^{T}$。

\begin{lstlisting}[language=matlab,title={computeWithinClassScatter.m}]
function Sw = computeWithinClassScatter(blueData, redData)
    covBlue = cov(blueData);
    covRed = cov(redData);
    Sw = covBlue + covRed;
end
\end{lstlisting}

通过拉格朗日乘子法可以求得所求直线的方向向量:

\[
{\mathbf{\theta }}^{ * } = {\mathbf{S}}_{w}^{-1}\left( {{\mathbf{\mu }}_{1} - {\mathbf{\mu }}_{2}}\right)
\]

\begin{lstlisting}[language=matlab,title={computeLDAProjection.m}]
function w = computeLDAProjection(Sw, meanBlue, meanRed)
    w = inv(Sw) * (meanBlue' - meanRed');
end
\end{lstlisting}

于是不难写出代码:

\begin{lstlisting}[language=matlab,title={main.m}]
clear;

blueData = load('../exp5/ex3blue.dat');
redData = load('../exp5/ex3red.dat');

plotData(blueData, redData);
[meanBlue, meanRed] = computeMeans(blueData, redData);
Sw = computeWithinClassScatter(blueData, redData);
w = computeLDAProjection(Sw, meanBlue, meanRed);
plotProjectionLine(w, blueData, redData);
plotPerpendiculars(blueData, redData, w);
\end{lstlisting}

其中绘图部分的代码为:

\begin{lstlisting}[language=matlab,title={plotProjectionLine.m}]
function plotProjectionLine(w, blueData, redData)
    k = w(2) / w(1);
    xMin = min([blueData(:, 1); redData(:, 1)]) - 1;
    xMax = max([blueData(:, 1); redData(:, 1)]) + 1;
    xRange = linspace(xMin, xMax * sqrt(2), 100);
    yRange = k * xRange;
    plot(xRange, yRange, 'k--', 'LineWidth', 1.5);
    legend({'Blue Point', 'Red Point', 'Projection Line'}, 'Location', 'northeast');
end
\end{lstlisting}

\begin{lstlisting}[language=matlab,title={plotPerpendiculars.m}]
function plotPerpendiculars(blueData, redData, w)
    k = w(2) / w(1);
    for i = 1:size(blueData, 1)
        [xFoot, yFoot] = findFoot(blueData(i, :), k);
        plot([blueData(i, 1), xFoot], [blueData(i, 2), yFoot], '--b', 'HandleVisibility', 'off');
        plot(xFoot, yFoot, 'ob', 'HandleVisibility', 'off');
    end
    for i = 1:size(redData, 1)
        [xFoot, yFoot] = findFoot(redData(i, :), k);
        plot([redData(i, 1), xFoot], [redData(i, 2), yFoot], '--r', 'HandleVisibility', 'off');
        plot(xFoot, yFoot, 'or', 'HandleVisibility', 'off');
    end
end

\end{lstlisting}

可以得到结果

\begin{figure}[H]
    \centering
    \includesvg[width=10cm]{imgs/5-1-2.svg}
\end{figure} 

\section{Experiment 5.2 LDA for N Classes}

\subsection{预处理数据}

首先读取数据:

\begin{lstlisting}[language=matlab,title={读入 ex3?.dat}]
ex3blue = load('../exp5/ex3blue.dat');
ex3green = load('../exp5/ex3green.dat');
ex3red = load('../exp5/ex3red.dat');
\end{lstlisting}

\subsection{Linear Discriminant Analysis- C Classes}

对于 $C$ 类的情况,可能有 $p\leq C-1$ 个 projection vectors:

\[
{\mathbf{S}}_{w}^{-1}{\mathbf{S}}_{b}{\mathbf{\theta }}_{i} = \lambda {\mathbf{\theta }}_{i},i = 1,2,\ldots ,p
\]

${{\Theta }}^{ * }$ 的列是对应于最大特征值的特征向量:

\[
\begin{matrix} {\mathbf{S}}_{w}^{-1}{\mathbf{S}}_{b}{{\Theta }}^{ * } = \lambda {{\Theta }}^{ * } & & & {{\Theta }}^{ * } = \left\lbrack  {{\mathbf{\theta }}_{1}^{ * },{\mathbf{\theta }}_{2}^{ * },...,{\mathbf{\theta }}_{p}^{ * }}\right\rbrack   \end{matrix}
\]

实际计算过程与 2 Classes 类似,但计算类内散度矩阵、计算类间散度矩阵的方式有变化:

对于类内散度矩阵,有:

\[
\begin{matrix} {\mathbf{S}}_{w} = \mathop{\sum }\limits_{{i = 1}}^{C}{\mathbf{S}}_{wi} & & {\mathbf{S}}_{wi} = \mathop{\sum }\limits_{{\mathbf{x} \in  \mathbf{{C}_{i}}}}\left( {\mathbf{x} - {\mathbf{\mu }}_{i}}\right) {\left( \mathbf{x} - {\mathbf{\mu }}_{i}\right) }^{T} & & & {\mathbf{S}}_{w} \in  {\mathbb{R}}^{d \times  d} \end{matrix}
\]

\begin{lstlisting}[language=matlab,title={computeWithinClassScatter.m}]
function Sw = computeWithinClassScatter(X, y)
    uniqueClasses = unique(y);
    Sw = zeros(size(X,2), size(X,2));
    for i = 1:length(uniqueClasses)
        class_idx = (y == uniqueClasses(i));
        Xi = X(class_idx, :);
        mu_i = mean(Xi);
        Xi = Xi - mu_i;
        Sw = Sw + Xi' * Xi;
    end
end
\end{lstlisting}

对于类间散度矩阵,有:

\[
\begin{array}{lll} {\mathbf{S}}_{b} = \mathop{\sum }\limits_{{i = 1}}^{C}{n}_{i}\left( {{\mathbf{\mu }}_{i} - \mathbf{\mu }}\right) {\left( {\mathbf{\mu }}_{i} - \mathbf{\mu }\right) }^{T} = \frac{1}{2N}\mathop{\sum }\limits_{{i,j = 1}}^{C}{n}_{i}{n}_{j}\left( {{\mathbf{\mu }}_{i} - {\mathbf{\mu }}_{j}}\right) {\left( {\mathbf{\mu }}_{i} - {\mathbf{\mu }}_{j}\right) }^{T} & & {\mathbf{S}}_{b} \in  {\mathbb{R}}^{d \times  d} \end{array}
\]

\begin{lstlisting}[language=matlab,title={computeBetweenClassScatter.m}]
function Sb = computeBetweenClassScatter(X, y)
    mu = mean(X);
    Sb = zeros(size(X,2), size(X,2));
    uniqueClasses = unique(y);
    for i = 1:length(uniqueClasses)
        class_idx = (y == uniqueClasses(i));
        Xi = X(class_idx, :);
        ni = size(Xi, 1);
        mu_i = mean(Xi);
        mu_diff = mu_i - mu;
        Sb = Sb + ni * (mu_diff' * mu_diff);
    end
end
\end{lstlisting}

可以得到结果:

\begin{figure}[H]
    \centering
    \includesvg[width=10cm]{imgs/5-1-3.svg}
\end{figure} 

\subsection{完整代码}

\begin{lstlisting}[language=matlab,title={main.m}]
clear;

ex3blue = load('../exp5/ex3blue.dat');
ex3green = load('../exp5/ex3green.dat');
ex3red = load('../exp5/ex3red.dat');
X = [ex3blue; ex3green; ex3red];
y = [ones(size(ex3blue,1),1); 2*ones(size(ex3green,1),1); 3*ones(size(ex3red,1),1)];

Sw = computeWithinClassScatter(X, y);
Sb = computeBetweenClassScatter(X, y);

[~, ~, V] = svd(inv(Sw) * Sb);
w = V(:, 1);
k = w(2) / w(1);

xRange = [0, max(10, max(X(:, 1)))];
yRange = k * xRange;

figure;
hold on;
scatter(ex3blue(:,1), ex3blue(:,2), 'b', 'filled', 'DisplayName', 'Blue Point');
scatter(ex3green(:,1), ex3green(:,2), 'g', 'filled', 'DisplayName', 'Green Point');
scatter(ex3red(:,1), ex3red(:,2), 'r', 'filled', 'DisplayName', 'Red Point');
plot(xRange, yRange, 'k--', 'LineWidth', 2, 'DisplayName', 'Projection Line');
legend;

axis equal;
xlabel('X');
ylabel('Y');
title('Scatter Plot with Projection Line');

plotPerpendiculars(ex3blue, w, 'b');
plotPerpendiculars(ex3green, w, 'g');
plotPerpendiculars(ex3red, w, 'r');

hold off;
\end{lstlisting}



\newpage
	% 代码分析:模块功能、涉及到的类、类关系、数据结构及关键代码等;
	% 任务要求,设计任务要求;
	% 设计:详细的设计方案,相关的数据结构、算法描述,可采用伪代码等形式化描述
	% 实现:修改哪些类、如何修改、为什么修改等;
	% 测试:测试用例,测试结果及结果分析,测试运行界面等;
	% 调试:调试方法,遇到的问题及解决方案等;
	% 结论与展望:完成的主要工作、收获、进一步的工作,建议、体会、心得等;

\section{Experiment 7: SVM}

\subsection{SVM}

支持向量机是一种监督学习模型,广泛应用于分类和回归问题。其核心思想是通过一个超平面将不同类别的样本数据进行分隔,从而实现分类。在二分类问题中,假设数据集为\(\{(x_i, y_i)\}\),其中\(x_i \in \mathbb{R}^n\)是特征向量,\(y_i \in \{-1, +1\}\)是类别标签。SVM的目标是找到一个决策超平面,该平面可以将两类数据分开,并且尽可能使得两类数据之间的间隔(即“边界”)最大化。

一个超平面可以表示为

\[
w \cdot x + b = 0
\]

其中,\(w\)是法向量,决定了超平面的方向,\(b\)是偏置,控制超平面的位置。

SVM的目标是最大化两类数据点到超平面的间隔。这个间隔可以通过支持向量(离超平面最近的样本点)来定义。为了最大化间隔,我们需要优化以下目标:

\[
\min_{w, b} \frac{1}{2} \|w\|^2
\]

同时满足约束条件:

\[
y_i (w \cdot x_i + b) \geq 1, \quad \forall i
\]

这意味着每个样本点都被正确分类,并且位于超平面的一侧。


首先读取数据:

\begin{lstlisting}[language=matlab,title={main.m}]
data1 = load('training_1.txt');
X1 = data1(:, 1:2);
y1 = data1(:, 3);

test1 = load('test_1.txt');
X_test1 = test1(:, 1:2);
y_test1 = test1(:, 3);
\end{lstlisting}

SVM 的训练:

\begin{lstlisting}[language=matlab,title={trainSVM.m}]
function [C, w, b] = trainSVM(X, y, C)
    H = (y * y') .* (X * X');
    f = -ones(size(y));
    A = [];
    b = [];
    Aeq = y';
    beq = 0;
    lb = zeros(size(y));
    ub = 0.00001 * ones(size(y));
    
    alpha = quadprog(H, f, A, b, Aeq, beq, lb, ub);

    w = sum((alpha .* y) .* X)';
    b = mean(y - X * w);
end
\end{lstlisting}

SVM 的测试:

\begin{lstlisting}[language=matlab,title={testSVM.m}]
function misclassified_fraction = testSVM(X_test, y_test, w, b)
    predictions = X_test * w + b;
    predicted_labels = sign(predictions);
    
    misclassified = sum(predicted_labels ~= y_test) / length(y_test);
    misclassified_fraction = misclassified;
end
\end{lstlisting}

不难画出训练集与测试集上的图像:

\begin{lstlisting}[language=matlab,title={plotDecisionBoundary.m}]
function plotDecisionBoundary(X, y, w, b, title_text)
    % Function to plot decision boundary and data points
    figure;
    x1_range = linspace(min(X(:, 1)), max(X(:, 1)), 100);
    x2_range = linspace(min(X(:, 2)), max(X(:, 2)), 100);
    [x1_grid, x2_grid] = meshgrid(x1_range, x2_range);
    X_grid = [x1_grid(:), x2_grid(:)];
    predictions_grid = X_grid * w + b;
    
    contour(x1_grid, x2_grid, reshape(predictions_grid, size(x1_grid)), [0, 0], 'k', 'LineWidth', 2);
    hold on;
    scatter(X(:, 1), X(:, 2), 50, y, 'filled');
    title(title_text);
    xlabel('x1');
    ylabel('x2');
    legend('Decision Boundary', 'Class 1', 'Class -1');
end
\end{lstlisting}

\begin{figure}[H]
    \centering
    \includesvg[width=10cm]{imgs/7-1-1.svg}
\end{figure} 

\begin{figure}[H]
    \centering
    \includesvg[width=10cm]{imgs/7-1-2.svg}
\end{figure} 

\subsection{Handwritten Digit Recognition}

首先参考给出的 strimage.m 读入文件:

\begin{lstlisting}[language=matlab,title={main.m}]
acc = [];
X = [];
y = [];

fidin = fopen('train-01-images.svm');
i = 1;
apres = [];
while ~feof(fidin)
    tline = fgetl(fidin);
    apres{i} = tline;
    i = i + 1;
end

for i = 1:12665
    a = char(apres(i));
    lena = size(a, 2);
    xy = sscanf(a(4:lena), '%d:%d');
    lenxy = size(xy, 1);
    
    grid = zeros(1, 784);
    for j = 2:2:lenxy
        if (xy(j) <= 0)
            break
        end
        grid(xy(j-1)) = xy(j) * 100 / 255;  % Scale pixel value
    end
    
    X = [X; grid];
    if startsWith(a, '+')
        y = [y, 1];
    elseif startsWith(a, '-')
        y = [y, -1];
    end
end
fclose(fidin);

Cval = [0.0001, 0.001, 1, 10, 100, 1000, 10000, 1000000, 100000000];
\end{lstlisting}

训练代码为:

\begin{lstlisting}[language=matlab,title={main.m}]
for i = 1:length(Cval)
    C = Cval(i);
    [w, b] = trainSVM(X, y', C);
    
    train_error = computeError(X, y', w, b);
    fprintf('C = %.7f, Training Error: %.4f\n', C, train_error);
    
    fidin = fopen('test-01-images.svm');
    apres = [];
    while ~feof(fidin)
        tline = fgetl(fidin);
        apres{i} = tline;
        i = i + 1;
    end
    X_test = [];
    y_test = [];
    
    for j = 1:2115
        a = char(apres(j));
        lena = size(a, 2);
        xy = sscanf(a(4:lena), '%d:%d');
        lenxy = size(xy, 1);
        
        grid = zeros(1, 784);  % Initialize feature vector for test image
        for k = 2:2:lenxy
            if (xy(k) <= 0)
                break
            end
            grid(xy(k-1)) = xy(k) * 100 / 255;  % Scale pixel value
        end
        
        X_test = [X_test; grid];
        if startsWith(a, '+')
            y_test = [y_test, 1];
        elseif startsWith(a, '-')
            y_test = [y_test, -1];
        end
    end
    fclose(fidin);
    
    test_error = computeError(X_test, y_test', w, b);
    fprintf('C = %.7f, Test Error: %.4f\n', C, test_error);

    acc = [acc, test_error];
end
\end{lstlisting}

可以得到收敛结果:

\begin{lstlisting}[language=matlab,title={main.m}]
Optimization completed: The relative dual feasibility, 6.508655e-12,
is less than options.OptimalityTolerance = 1.000000e-08, the complementarity measure,
1.205339e-09, is less than options.OptimalityTolerance, and the relative maximum constraint
violation, 1.100637e-18, is less than options.ConstraintTolerance = 1.000000e-08.
\end{lstlisting}

可以找出无法分类的图像:

\begin{lstlisting}[language=matlab,title={main.m}]
misclassified_idx = findMisclassified(X_train, y_train, w0, b0);
visualizeMisclassified(X_train, misclassified_idx);
\end{lstlisting}

\begin{lstlisting}[language=matlab,title={findMisclassified.m}]
function misclassified_idx = findMisclassified(X, y, w, b)
    % Function to find the index of a misclassified example
    predictions = X * w + b;
    predicted_labels = sign(predictions);
    misclassified_idx = find(predicted_labels ~= y, 1, 'first'); % Find the first misclassified example
end
\end{lstlisting}

\begin{lstlisting}[language=matlab,title={visualizeMisclassified.m}]
function visualizeMisclassified(X, idx)
    % Function to visualize a misclassified example
    misclassified_image = reshape(X(idx, :), [28, 28]); % Reshape to 28x28 image
    figure;
    imshow(misclassified_image, []);
    title('Misclassified Image');
end
\end{lstlisting}

\begin{figure}[H]
    \centering
    \includesvg[width=5cm]{imgs/7-2-1.svg}
\end{figure} 


\subsection{Non-Linear SVM}

进行升维:

\begin{lstlisting}[language=matlab,title={trainNonLinearSVM.m}]
            K(i, j) = exp(-r * norm(X(i, :) - X(j, :))^2);
\end{lstlisting}

不难写出训练函数:

\begin{lstlisting}[language=matlab,title={trainNonLinearSVM.m}]
function [alpha, w, b] = trainNonLinearSVM(X, y, C, r)
    % Train a non-linear SVM using the RBF kernel with specified regularization parameter C and gamma (r)
    n = size(X, 1);
    
    % Compute the kernel matrix K
    K = zeros(n, n);
    for i = 1:n
        for j = 1:n
            K(i, j) = exp(-r * norm(X(i, :) - X(j, :))^2);
        end
    end
    
    % Set up the optimization problem for SVM
    H = (y * y') .* K;
    f = -ones(n, 1);
    A = [];
    b = [];
    Aeq = y';
    beq = 0;
    lb = zeros(n, 1);
    ub = C * ones(n, 1);
    
    % Solve the quadratic programming problem
    alpha = quadprog(H, f, A, b, Aeq, beq, lb, ub);
    
    % Calculate the weight vector w and bias b
    w = sum((alpha .* y) .* X, 1)';
    b = mean(y - K * (alpha .* y));
end
\end{lstlisting}


\begin{figure}[H]
    \centering
    \includesvg[width=18cm]{imgs/7-3-1.svg}
\end{figure} 

\subsection{完整代码}

\begin{lstlisting}[language=matlab,title={SVM - main.m}]
clear;
data1 = load('training_1.txt');
X1 = data1(:, 1:2);
y1 = data1(:, 3);
test1 = load('test_1.txt');
X_test1 = test1(:, 1:2);
y_test1 = test1(:, 3);
[C, w, b] = trainSVM(X1, y1, 0.00001); % Regularization term C
plotDecisionBoundary(X1, y1, w, b, 'Dataset 1: Decision Boundary');
misclassified_fraction = testSVM(X_test1, y_test1, w, b);
fprintf('Fraction of misclassified examples for Dataset 1: %.2f\n', misclassified_fraction);
data2 = load('training_2.txt');
X2 = data2(:, 1:2);
y2 = data2(:, 3);
test2 = load('test_2.txt');
X_test2 = test2(:, 1:2);
y_test2 = test2(:, 3);
[C2, w2, b2] = trainSVM(X2, y2, 0.00001);
plotDecisionBoundary(X2, y2, w2, b2, 'Dataset 2: Decision Boundary');
misclassified_fraction2 = testSVM(X_test2, y_test2, w2, b2);
fprintf('Fraction of misclassified examples for Dataset 2: %.2f\n', misclassified_fraction2);
\end{lstlisting}


\begin{lstlisting}[language=matlab,title={plotDecisionBoundary.m}]
function plotDecisionBoundary(X, y, alpha, r, C)
    % This function plots the decision boundary for a non-linear SVM with RBF kernel.
    % X: Training data (n x 2)
    % y: Labels (n x 1)
    % alpha: Coefficients from SVM solution (n x 1)
    % r: Gamma parameter for RBF kernel
    % C: Regularization parameter

    % Create a grid for plotting the decision boundary
    [X1, X2] = meshgrid(linspace(min(X(:, 1)), max(X(:, 1)), 100), ...
                        linspace(min(X(:, 2)), max(X(:, 2)), 100));
    grid_points = [X1(:), X2(:)];
    
    % Compute the kernel matrix for grid points
    K_test = zeros(size(grid_points, 1), size(X, 1));
    for i = 1:size(grid_points, 1)
        for j = 1:size(X, 1)
            K_test(i, j) = exp(-r * norm(grid_points(i, :) - X(j, :))^2);
        end
    end
    
    % Compute decision values
    decisions = K_test * (alpha .* y);
    
    % Plot decision boundary
    contour(X1, X2, reshape(decisions, size(X1)), [0, 0], 'LineWidth', 2);
    hold on;
    
    % Scatter plot for training data
    scatter(X(:, 1), X(:, 2), 50, y, 'filled');
    hold off;
    
    % Labels
    xlabel('x_1');
    ylabel('x_2');
    title(['Decision Boundary: C = ', num2str(C), ', r = ', num2str(r)]);
    colorbar;
end
\end{lstlisting}


\begin{lstlisting}[language=matlab,title={loadImageData.m}]
function [X, y] = loadImageData(filename)
    % 读取SVM格式的图像数据(每个图像数据为一行)
    fidin = fopen(filename); % 打开文件
    i = 1;
    apres = {};

    % 读取每一行数据
    while ~feof(fidin)
        tline = fgetl(fidin); % 从文件读取一行
        apres{i} = tline;
        i = i + 1;
    end

    fclose(fidin);

    % 初始化数据矩阵
    X = [];
    y = [];
    
    % 遍历读取的每一行
    for n = 1:length(apres)
        a = char(apres{n});
        
        % 获取标签
        label = str2double(a(1));
        y = [y; label];
        
        % 获取特征数据 (每个图像为784维)
        lena = length(a);
        xy = sscanf(a(4:lena), '%d:%d'); % 解析特征
        
        % 初始化一个784维的特征向量
        grid = zeros(1, 784); 
        lenxy = length(xy) / 2;
        
        % 填充特征数据到特征向量
        for i = 1:2:lenxy
            index = xy(i); % 特征索引
            value = xy(i+1); % 特征值
            if index > 0
                grid(index) = value * 100 / 255; % 归一化处理
            end
        end
        
        % 将特征向量添加到特征矩阵
        X = [X; grid];
    end
    
    % 转换为double类型
    X = double(X);
    y = double(y);
end
\end{lstlisting}


\begin{lstlisting}[language=matlab,title={trainNonLinearSVM.m}]
[X_train, y_train] = loadData('training_3.text');

C_values = [1, 100, 1000];
r_values = [0.1, 10, 100];

figure;
hold on;

for i = 1:length(C_values)
    for j = 1:length(r_values)
        C = C_values(i);
        r = r_values(j);
        
        [alpha, w, b] = trainNonLinearSVM(X_train, y_train, C, r);
        
        subplot(length(C_values), length(r_values), (i-1) * length(r_values) + j);
        plotDecisionBoundary(X_train, y_train, alpha, r, C);
        
        title(['C = ' num2str(C) ', r = ' num2str(r)]);
    end
end

xlabel('x_1');
ylabel('x_2');
sgtitle('Non-linear SVM Decision Boundaries');
grid on;
hold off;
\end{lstlisting}

\end{document} 


