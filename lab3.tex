	% 代码分析:模块功能、涉及到的类、类关系、数据结构及关键代码等;
	% 任务要求,设计任务要求;
	% 设计:详细的设计方案,相关的数据结构、算法描述,可采用伪代码等形式化描述
	% 实现:修改哪些类、如何修改、为什么修改等;
	% 测试:测试用例,测试结果及结果分析,测试运行界面等;
	% 调试:调试方法,遇到的问题及解决方案等;
	% 结论与展望:完成的主要工作、收获、进一步的工作,建议、体会、心得等;

\section{Experiment 3: Logistic Regression and Newton's Method}

\subsection{预处理数据}

首先读取数据:

\begin{lstlisting}[language=matlab,title={读入 ex4?.dat}]
x = load('ex4x.dat');
y = load('ex4y.dat');

[m, n] = size(x);
x = [ones(m, 1), x];
\end{lstlisting}

\subsection{输入数据可视化}

\begin{lstlisting}[language=matlab,title={绘制图像}]
pos = find(y == 1);
neg = find(y == 0);

figure; hold on;
plot(x(pos, 2), x(pos, 3), 'k+', 'LineWidth', 2, 'MarkerSize', 7);
plot(x(neg, 2), x(neg, 3), 'ko', 'MarkerFaceColor', 'y', 'MarkerSize', 7);
xlabel('考试1成绩');
ylabel('考试2成绩');
legend('已录取', '未录取');
hold off;
\end{lstlisting}

\begin{figure}[H]
    \centering
    \includesvg[width=10cm]{imgs/3-1.svg}
\end{figure} 

\subsection{牛顿法}

这里我将 g 写成了一个函数:

\begin{lstlisting}[language=matlab,title={sigmoid.m}]
function g = sigmoid(z)
    g = 1.0 ./ (1.0 + exp(-z));
end
\end{lstlisting}

可以对着写出 $J(\theta)$ 的函数:

\[
J\left( \theta \right)  = \frac{1}{m}\mathop{\sum }\limits_{{i = 1}}^{m}\left\lbrack  {-{y}^{\left( i\right) }\log \left( {{h}_{\theta }\left( {x}^{\left( i\right) }\right) }\right)  - \left( {1 - {y}^{\left( i\right) }}\right) \log \left( {1 - {h}_{\theta }\left( {x}^{\left( i\right) }\right) }\right) }\right\rbrack
\]

\begin{lstlisting}[language=matlab,title={costFunction.m}]
function [J, grad] = costFunction(theta, x, y)
    m = length(y);
    h = sigmoid(x * theta);
    
    J = (1/m) * sum(-y .* log(h) - (1 - y) .* log(1 - h));
    grad = (1/m) * (x' * (h - y));
end

\end{lstlisting}

Hessian 矩阵为:

\[
{\nabla }_{\theta }J = \frac{1}{m}\mathop{\sum }\limits_{{i = 1}}^{m}\left( {{h}_{\theta }\left( {x}^{\left( i\right) }\right)  - {y}^{\left( i\right) }}\right) {x}^{\left( i\right) }
\]

故可以在牛顿法写出:

\[
{\theta }^{\left( t + 1\right) } = {\theta }^{\left( t\right) } - {H}^{-1}{\nabla }_{\theta }J
\]

\begin{lstlisting}[language=matlab,title={newtonsMethod.m}]
function [theta, J_history] = newtonsMethod(x, y, theta, max_iters)
    m = length(y);
    J_history = zeros(max_iters, 1);

    for iter = 1:max_iters
        h = sigmoid(x * theta);
        grad = (1/m) * (x' * (h - y));
        H = (1/m) * (x' * diag(h .* (1 - h)) * x);
        theta = theta - H\grad;

        J_history(iter) = (1/m) * sum(-y .* log(h) - (1 - y) .* log(1 - h));
    end
end
\end{lstlisting}

然后可以画出决策边界:

\begin{lstlisting}[language=matlab,title={plotDecisionBoundary.m}]
function plotDecisionBoundary(theta, x, y)
    pos = find(y == 1);
    neg = find(y == 0);
    
    figure; hold on;
    plot(x(pos, 2), x(pos, 3), 'k+', 'LineWidth', 2, 'MarkerSize', 7);
    plot(x(neg, 2), x(neg, 3), 'ko', 'MarkerFaceColor', 'y', 'MarkerSize', 7);
    
    plot_x = [min(x(:,2))-2, max(x(:,2))+2];
    plot_y = (-1./theta(3)) .* (theta(2).*plot_x + theta(1));
    
    plot(plot_x, plot_y, 'b-', 'LineWidth', 2);
    
    xlabel('考试1成绩');
    ylabel('考试2成绩');
    legend('已录取', '未录取', '决策边界');
    hold off;
end
\end{lstlisting}

\begin{figure}[H]
    \centering
    \includesvg[width=10cm]{imgs/3-2.svg}
\end{figure} 

\subsection{questions}

\begin{lstlisting}[language=c++,title={4.1 - 问题描述}]
What values of theta did you get? How many iterations were required for convergence?(你得到的 theta 值是多少?需要多少次迭代才能收敛?)
\end{lstlisting}

\begin{lstlisting}[language=matlab,title={4.1 - 求 theta 值与观察收敛速度}]
figure;
plot(1:max_iters, J_history(1:max_iters), '-b', 'LineWidth', 2);
xlabel('迭代次数');
ylabel('成本函数 J(theta)');
title('成本函数随迭代次数的变化');

fprintf('优化后的 theta 值为:\n');
disp(theta);
fprintf('收敛所需的迭代次数:%d\n', max_iters);
\end{lstlisting}

得到值为:

\begin{lstlisting}[language=matlab,title={4.1 - $\theta$ 值}]
优化后的 theta 值为:
  -16.3787
    0.1483
    0.1589
\end{lstlisting}

以及对应的函数图像:

\begin{figure}[H]
    \centering
    \includesvg[width=10cm]{imgs/3-3.svg}
\end{figure} 

可以看出,在 5 次迭代时就已收敛。

\begin{lstlisting}[language=c++,title={4.2 - 问题描述}]
What is the probability that a student with a score of 20 on Exam 1 and a score of 80 on Exam 2 will not be admitted?(考试 1 的成绩为 20 分,考试 2 的成绩为 80 分的学生不被录取的概率是多少?)
\end{lstlisting}

\begin{lstlisting}[language=matlab,title={4.2 - 预测概率}]
student_scores = [1, 20, 80];
prob = sigmoid(student_scores * theta);

fprintf('不被录取的概率: %f\n', prob);
\end{lstlisting}

得到结果:

\begin{lstlisting}[language=matlab,title={预测概率}]
不被录取的概率: 0.331978
\end{lstlisting}

\subsection{完整代码}

\begin{lstlisting}[language=matlab,title={完整代码}]
clear; clc;

x = load('ex4x.dat');
y = load('ex4y.dat');

[m, n] = size(x);
x = [ones(m, 1), x];

pos = find(y == 1);
neg = find(y == 0);

figure; hold on;
plot(x(pos, 2), x(pos, 3), 'k+', 'LineWidth', 2, 'MarkerSize', 7);
plot(x(neg, 2), x(neg, 3), 'ko', 'MarkerFaceColor', 'y', 'MarkerSize', 7);
xlabel('考试1成绩');
ylabel('考试2成绩');
legend('已录取', '未录取');
hold off;

initial_theta = zeros(n + 1, 1);
max_iters = 15;
[theta, J_history] = newtonsMethod(x, y, initial_theta, max_iters);

% 绘制决策边界
plotDecisionBoundary(theta, x, y);

figure;
plot(1:max_iters, J_history(1:max_iters), '-b', 'LineWidth', 2);
xlabel('迭代次数');
ylabel('成本函数 J(theta)');
title('成本函数随迭代次数的变化');

fprintf('优化后的 theta 值为:\n');
disp(theta);
fprintf('收敛所需的迭代次数:%d\n', max_iters);

student_scores = [1, 20, 80];
prob = sigmoid(student_scores * theta);

fprintf('不被录取的概率: %f\n', prob);
\end{lstlisting}