	% 代码分析:模块功能、涉及到的类、类关系、数据结构及关键代码等;
	% 任务要求,设计任务要求;
	% 设计:详细的设计方案,相关的数据结构、算法描述,可采用伪代码等形式化描述
	% 实现:修改哪些类、如何修改、为什么修改等;
	% 测试:测试用例,测试结果及结果分析,测试运行界面等;
	% 调试:调试方法,遇到的问题及解决方案等;
	% 结论与展望:完成的主要工作、收获、进一步的工作,建议、体会、心得等;

\section{Experiment 2: Multivariate Linear Regression}

\subsection{预处理数据}

首先读取数据:

\begin{lstlisting}[language=matlab,title={读入 ex2?.dat}]
x = load('ex2x.dat');
y = load('ex2y.dat');
m = length(y);
x = [ones(m, 1), x];
\end{lstlisting}

对数据进行特征缩放,也就是进行标准化,公式为:

\[
x_{i,\text{scaled}}=\frac{x_i-\mu_i}{\sigma_i}
\]

其中 $\mu_i$ 为均值、$\sigma_i$ 为标准差,在 matlab 中有函数 mean() 与 std() 可以计算,如此可让不同数量级的特征位于统一尺度下、加速收敛。

\begin{lstlisting}[language=matlab,title={标准化 x}]
sigma = std(x);
mu = mean(x);
x(:, 2) = (x(:, 2) - mu(2)) / sigma(2);
x(:, 3) = (x(:, 3) - mu(3)) / sigma(3);
\end{lstlisting}

\subsection{$ J(\mathbf \theta )$ 的计算}

$J(\mathbf \theta ) $ 为代价函数,表示迭代的代价。计算公式如下:

\begin{align}
    h_{\mathbf \theta }(\mathbf x)&=\mathbf \theta ^T\mathbf x=\sum_{i = 0}^{n} \theta _ix_i\\
    \theta _j &=\theta _j-\alpha \frac{1}{m}\sum_{i = 1}^{m}  (h_{\theta } (x^{(i)})-y^{(i)})x_j ^i\\
    J(\mathbf \theta )&=\frac{1}{2m} (\mathbf X\mathbf \theta -\mathbf y)^T(\mathbf X\mathbf \theta -\mathbf y)
\end{align}

于是不难写出代码(加入 J\_history 记录 $J(\mathbf \theta ) $ 的值方便追踪过程):

\begin{lstlisting}[language=matlab,title={gradientDescent.m}]
function [theta, J_history] = gradientDescent(x, y, theta, alpha, iterations)
    m = length(y);
    J_history = zeros(iterations, 1);
    
    for i = 1:iterations
        error = x * theta - y;
        J_history(i) = (error' * error) / (2 * m);
        
        gradient = (x' * error) / m;
        theta = theta - alpha * gradient;
    end
end
\end{lstlisting}

\subsection{选择学习率}

首先设置学习参数:

\begin{lstlisting}[language=matlab,title={梯度下降参数}]
iterations = 50;
theta_init = zeros(size(x, 2), 1);
\end{lstlisting}

可用 gradientDescent 测试不同学习率下收敛速率:

\begin{lstlisting}[language=matlab,title={对比不同学习率}]
[theta1, J1] = gradientDescent(x, y, theta_init, 1, iterations);
[theta2, J2] = gradientDescent(x, y, theta_init, 0.2, iterations);
[theta3, J3] = gradientDescent(x, y, theta_init, 0.04, iterations);
\end{lstlisting}

绘图,得:

\begin{figure}[H]
    \centering
    \includesvg[width=10cm]{imgs/gradientDescent.svg}
\end{figure} 

\begin{lstlisting}[language=matlab,title={画图}]
figure;
plot(0:iterations-1, J1, 'r-', 'DisplayName', 'alpha = 1');
hold on;
plot(0:iterations-1, J2, 'b-', 'DisplayName', 'alpha = 0.2');
plot(0:iterations-1, J3, 'g-', 'DisplayName', 'alpha = 0.04');
legend;
xlabel('Iterations');
ylabel('Cost J');
hold off;
\end{lstlisting}

\subsection{questions}

\begin{lstlisting}[language=c++,title={5.1 - 问题描述}]
Observe the changes in the cost function happens as the learning rate changes. What happens when the learning rate is too small? Too large?
\end{lstlisting}

当 $\alpha$ 太小时,梯度下降收敛速度很慢,参数更新幅度很小,需要更多迭代才能接近最优解;当 $\alpha$ 太大时,梯度下降可能会越过最优解,导致 $J(\mathbf{\theta})$ 震荡,甚至无法收敛。

\begin{lstlisting}[language=c++,title={5.2 - 问题描述}]
Using the best learning rate that you found, run gradient descent until convergence to find
(a) The final values of theta
(b) The predicted price of a house with 1650 square feet and 3 bedrooms. Don’t forget to scale your features when you make this prediction!
\end{lstlisting}

\begin{enumerate}
    \item $J(\mathbf \theta) = 2.043\times 10^9, \mathbf \theta =[3.404\times 10^5,1.106\times 10^5,-6.625\times 10^3]$;
    \item 预测值为 $2.93\times 10^5$。
\end{enumerate}

\begin{lstlisting}[language=c++,title={6.1 - 问题描述}]
In your program, use the formula above to calculate theta. Remember that while you don’t need to scale your features, you still need to add an intercept term.
\end{lstlisting}

结果为$\theta =[8.960\times 10^4,1.392\times 10^2,-8.738\times 10^3]$。

\begin{lstlisting}[language=c++,title={6.2 - 问题描述}]
Once you have found theta from this method, use it to make a price prediction for a 1650-square-foot house with 3 bedrooms. Did you get the same price that you found through gradient descent?
\end{lstlisting}

结果为 $2.931\times 10^5$。

\subsection{完整代码}

gradientDescent.m 在上文中已给出

\begin{lstlisting}[language=matlab,title={完整代码}]
clear;
x = load('ex2x.dat');
y = load('ex2y.dat');
m = length(y);
x = [ones(m, 1), x];

% 特征缩放
sigma = std(x);
mu = mean(x);
x(:, 2) = (x(:, 2) - mu(2)) / sigma(2);
x(:, 3) = (x(:, 3) - mu(3)) / sigma(3);

% 梯度下降参数
iterations = 50;
theta_init = zeros(size(x, 2), 1);

% 不同学习率下的梯度下降
[theta1, J1] = gradientDescent(x, y, theta_init, 1, iterations);
[theta2, J2] = gradientDescent(x, y, theta_init, 0.2, iterations);
[theta3, J3] = gradientDescent(x, y, theta_init, 0.04, iterations);

% 绘图
figure;
plot(0:iterations-1, J1, 'r-', 'DisplayName', 'alpha = 1');
hold on;
plot(0:iterations-1, J2, 'b-', 'DisplayName', 'alpha = 0.2');
plot(0:iterations-1, J3, 'g-', 'DisplayName', 'alpha = 0.04');
legend;
xlabel('Iterations');
ylabel('Cost J');
hold off;
\end{lstlisting}